% eulerlagrange.tex
%
% Copyright (C) 2020 José A. Navarro Ramón <josea.navarro1@gmail.com>
%
\chapter{Ecuaciones de Euler-Lagrange}

\section{Principio de D'Alembert}
El Principio de d'Alembert tiene, sobre todo,  una importancia histórica. A partir
de él se dedujo por primera vez el \emph{lagrangiano}. Actualmente se deduce a partir del
Principio Variacional.

\subsection{Principio de D'Alembert estático}
El \emph{Principio de D'Alembert estático} dice que:
\begin{quote}
  La suma de \emph{fuerzas aplicadas}\footnotemark{} sobre un sistema, multiplicadas escalarmente
  por las  variaciones virtuales de las posiciones de las partículas del sistema, debe dar cero.
\end{quote}
\footnotetext{Fuerzas externas.}
\begin{equation}\label{eq:el-PD-estatico}
  \sum_i \vvv{F}_i^{(a)} \cdot \delta\vvv{r}_i = 0
\end{equation}

El término \emph{fuerzas aplicadas} hace referencia a las fuerzas que se aplican directamente sobre
las partículas del sistema. Como consecuencia de estas y de la existencia de \emph{ligaduras} que
restringen sus movimientos, surgen unas fuerzas de ligadura como reacción a las primeras.

Para habituarnos al significado de las magnitudes implicadas vamos a resolver un ejercicio
sencillo, que resolveremos primero mediante los Principios Fundamentales de la Dinámica de
Newton, y después mediante el Principio de D'Alembert estático.

Antes de mostrar el ejercicio podríamos preguntarnos qué utilidad tiene el Principio de
D'Alembert, si se pueden utilizar las leyes de Newton. Resulta que a partir de este Principio
se puede definir el \emph{lagrangiano}. De hecho así ocurrió históricamente. Además, hay sistemas
(no mecánicos) que no se pueden resolver mediante las leyes de Newton, a los que sí se puede
aplicar el \emph{lagrangiano}. Por tanto, este último concepto es más general que las citadas leyes
de Newton.

\subsubsection{Ejercicio}
Consideremos una barra de densidad uniforme de longitud $L$ y masa $M$, que está apoyada
entre el suelo y una pared, formando un ángulo $\theta$ con el suelo, ver
figura~\ref{fig:el-barra-geometria}.
Para simplificar el ejercicio supondremos que hay fricción de la barra con el suelo, pero
no con la pared. El peso se aplica en el centro de masas de la barra y, si consideramos
que la barra es homogénea, el peso estará aplicado en la mitad de la barra.
Las fuerzas se representan en la figura~\ref{fig:el-barra-fuerzas}.

\begin{figure}[ht]
  % Escala
  \def\scl{.75}
  \centering
  \begin{subfigure}[b]{.45\textwidth}
    % Barra
    \pgfmathsetmacro{\LBARRA}{5.5}
    \pgfmathsetmacro{\ANGBARRA}{30}
    \pgfmathsetmacro{\RADIOANGBARRA}{3.0}
    \pgfmathsetmacro{\LHORBARRA}{\LBARRA * cos(\ANGBARRA)}
    \pgfmathsetmacro{\LVERBARRA}{\LBARRA * sin(\ANGBARRA)}
    % Suelo
    \pgfmathsetmacro{\LSUELO}{\LHORBARRA + .8}
    \pgfmathsetmacro{\LPARED}{\LVERBARRA + .7}
    % Soporte
    \pgfmathsetmacro{\DISTSOPORTE}{.3}
    \pgfmathsetmacro{\LSOPORTE}{1.1}
    \pgfmathsetmacro{\ANGSOPORTE}{230}
    \pgfmathsetmacro{\NUMTOTHOR}{int(\LSUELO / \DISTSOPORTE)}
    \pgfmathsetmacro{\NUMTOTVER}{int(\LPARED / \DISTSOPORTE)}
    % Fondo
    \pgfmathsetmacro{\BGTOP}{3.8}
    \pgfmathsetmacro{\BGBOTTOM}{-.8}
    \pgfmathsetmacro{\BGLEFT}{-.8}
    \pgfmathsetmacro{\BGRIGHT}{6.0}
    %
    \centering
    \begin{tikzpicture}[%
      scale=\scl,
      baseline,
      punto barra/.style={fill=black, draw=black, radius=.5pt},
      angulo/.style={draw=green!50!black, fill=green!20},
      every node/.style={black,font=\small},
      barra/.style={line width=1.4pt, black},
      fuerza/.style={-{Latex[length=8pt,width=4.5pt]}, line width=1pt},
      background/.style={%
        line width=\bgborderwidth,
        draw=\bgbordercolor,
        fill=\bgcolor,
      },
      ]
      % COORDENADAS
      \coordinate (O) at (0,0);
      \coordinate (xsuelo) at ($(O) + (\LSUELO, 0)$);
      \coordinate (ypared) at ($(O) + (0,\LPARED)$);
      \coordinate (barra1) at ($(O) + (\LHORBARRA, 0)$);
      \coordinate (barra2) at ($(O) + (0,\LVERBARRA)$);
      % 
      \coordinate (btop) at (0,\BGTOP);
      \coordinate (bbottom) at (0,\BGBOTTOM);
      \coordinate (bleft) at (\BGLEFT,0);
      \coordinate (bright) at (\BGRIGHT,0);
      % DIBUJO
      % Ángulo
      \filldraw[angulo,-{Latex[round]}]
      (barra1) -- +(left:\RADIOANGBARRA em)
      arc [start angle=180, end angle=180-\ANGBARRA, radius=\RADIOANGBARRA em] -- cycle;
      \node[above left=-2.2pt and 10.5pt] at (barra1) {\footnotesize $\theta$};
      % Suelo
      \draw (O) -- (xsuelo);
      % Pared
      \draw (O) -- (ypared);
      % Barra
      \draw[barra] (barra1) -- node[above] {$L$} (barra2);
      \filldraw[punto barra] (barra1) circle;
      \filldraw[punto barra] (barra2) circle;
      % Soporte horizontal
      \foreach \xn in {1,2,...,\NUMTOTHOR}
      \draw[black!50, ultra thin] (\DISTSOPORTE * \xn, 0) -- +(\ANGSOPORTE:\LSOPORTE em);
      % Soporte vertical
      \foreach \yn in {1,2,...,\NUMTOTVER}
      \draw[black!50, ultra thin] (0,\DISTSOPORTE * \yn) -- +(\ANGSOPORTE:\LSOPORTE em);
      % YELLOW BACKGROUND
      \begin{scope}[on background layer]
        \node [background, fit= (bleft) (bright) (btop) (bbottom)] {};
      \end{scope}
    \end{tikzpicture}
    \caption{}
    \label{fig:el-barra-geometria}
  \end{subfigure}
  \hspace{2em}
  \begin{subfigure}[b]{.45\textwidth}
    % Barra
    \pgfmathsetmacro{\LBARRA}{5.5}
    \pgfmathsetmacro{\ANGBARRA}{30}
    \pgfmathsetmacro{\RADIOANGBARRA}{2.7}
    \pgfmathsetmacro{\LHORBARRA}{\LBARRA * cos(\ANGBARRA)}
    \pgfmathsetmacro{\LVERBARRA}{\LBARRA * sin(\ANGBARRA)}
    % Suelo
    \pgfmathsetmacro{\LSUELO}{\LHORBARRA + .8}
    \pgfmathsetmacro{\LPARED}{\LVERBARRA + .7}
    % Soporte
    \pgfmathsetmacro{\DISTSOPORTE}{.3}
    \pgfmathsetmacro{\LSOPORTE}{1.1}
    \pgfmathsetmacro{\ANGSOPORTE}{230}
    \pgfmathsetmacro{\NUMTOTHOR}{int(\LSUELO / \DISTSOPORTE)}
    \pgfmathsetmacro{\NUMTOTVER}{int(\LPARED / \DISTSOPORTE)}
    % Peso
    \pgfmathsetmacro{\LPESO}{2.6}
    % Fuerza de rozamiento
    \pgfmathsetmacro{\LFR}{4.0}
    % N1
    \pgfmathsetmacro{\LNSUELO}{\LPESO}
    % N2
    \pgfmathsetmacro{\LNPARED}{\LFR}
    % Fondo
    \pgfmathsetmacro{\BGTOP}{3.8}
    \pgfmathsetmacro{\BGBOTTOM}{-.8}
    \pgfmathsetmacro{\BGLEFT}{-.8}
    \pgfmathsetmacro{\BGRIGHT}{6.0}
    %
    \centering
    \begin{tikzpicture}[%
      scale=\scl,
      baseline,
      punto fuerza/.style={fill=green!60!black, draw=green!60!black, radius=1.1pt},
      angulo/.style={draw=green!50!black, fill=green!20},
      every node/.style={black,font=\small},
      barra/.style={line width=1.4pt, black},
      fuerza/.style={-{Latex[length=8pt,width=4.5pt]}, line width=1.4pt, green!60!black},
      background/.style={%
        line width=\bgborderwidth,
        draw=\bgbordercolor,
        fill=\bgcolor,
      },
      ]
      % COORDENADAS
      \coordinate (O) at (0,0);
      \coordinate (xsuelo) at ($(O) + (\LSUELO, 0)$);
      \coordinate (ypared) at ($(O) + (0,\LPARED)$);
      \coordinate (barra1) at ($(O) + (\LHORBARRA, 0)$);
      \coordinate (barra2) at ($(O) + (0,\LVERBARRA)$);
      \coordinate (peso) at ($(O) + (.5*\LHORBARRA,.5*\LVERBARRA)$);
      % 
      \coordinate (btop) at (0,\BGTOP);
      \coordinate (bbottom) at (0,\BGBOTTOM);
      \coordinate (bleft) at (\BGLEFT,0);
      \coordinate (bright) at (\BGRIGHT,0);
      % DIBUJO
      % Suelo
      \draw (O) -- (xsuelo);
      % Pared
      \draw (O) -- (ypared);
      % Barra
      \draw[barra] (barra1) -- (barra2);
      % Soporte horizontal
      \foreach \xn in {1,2,...,\NUMTOTHOR}
      \draw[black!50, ultra thin] (\DISTSOPORTE * \xn, 0) -- +(\ANGSOPORTE:\LSOPORTE em);
      % Soporte vertical
      \foreach \yn in {1,2,...,\NUMTOTVER}
      \draw[black!50, ultra thin] (0,\DISTSOPORTE * \yn) -- +(\ANGSOPORTE:\LSOPORTE em);
      % Peso
      \draw[fuerza] (peso) -- node[below left=0pt and 3pt] {$P$} +(down:\LPESO em);
      \filldraw[punto fuerza] (peso) circle;
      % fr
      \draw[fuerza] (barra1) -- +(left:\LFR em) node[above left=0pt and -8pt] {$f_r$};
      % N1
      \draw[fuerza] (barra1) -- node[above right=0pt and 0pt] {$N_1$} +(up:\LNSUELO em);
      \filldraw[punto fuerza] (barra1) circle;
      % N2
      \draw[fuerza] (barra2) -- node[above right=0pt and 0pt] {$N_2$} +(right:\LNPARED em);
      \filldraw[punto fuerza] (barra2) circle;
      % YELLOW BACKGROUND
      \begin{scope}[on background layer]
        \node [background, fit= (bleft) (bright) (btop) (bbottom)] {};
      \end{scope}
    \end{tikzpicture}
    \caption{}
    \label{fig:el-barra-fuerzas}
  \end{subfigure}
  \caption{Una barra de longitud $L$ está apoyada sobre una pared vertical que forma un
    ángulo $\theta$ con el suelo. En la imagen $(a)$ se considera la geometría del sistema,
    mientras que en la $(b)$ se representan las fuerzas, tanto aplicadas como de ligadura
    que actúan sobre ella.}
\end{figure}

Vamos a resolver el problema de dos formas:
\begin{itemize}
\item Aplicando las \emph{leyes de Newton}:
 
  La fuerza aplicada es el peso $P=Mg$. Debido a las características del problema, aparecerán
  dos fuerzas sobre la escalera que son perpendiculares a las superficies del suelo y de la pared,
  respectivamente. Estas se denominan fuerzas normales, $N_1$ y $N_2$.
  Por otro lado, en el suelo hay rozamiento, pues la barra tiende a deslizar sobre el suelo hacia
  la derecha, por lo que esta fuerza de rozamiento intentará contrarrestar dicho deslizamiento
  dirigiéndose hacia la izquierda.
  Recordamos que, por sencillez, no hay fricción de la barra con la pared.
  
  Como la barra está en equilibrio, la resultante de las fuerzas que actúan sobre ella es nula.
  \[
    \sum_i \vvv{F}_i = \vvv{0}
  \]
  \[
    \vvv{N}_2 + \vvv{P} + \vvv{N}_1 + \vvv{f}_r = \vvv{0}
  \]
  \[
    (N_2, 0) + (0,-mg) + (0,N_1) + (-f_r, 0) = (0, 0)
  \]

  De donde se deducen dos ecuaciones que relacionan los módulos de las fuerzas.
  \begin{align}\label{eq:el-ej-fuerzas-n2}
    f_r &= N_2\\
    \label{eq:el-ej-fuerzas-n1}
    N_1 &= mg
  \end{align}

  Además, la suma de momentos con respecto a un punto arbitrario, llamado (\emph{centro}),
  se anula.
  El momento de una fuerza con respecto a un punto es el producto vectorial.
  \[
    \vvv{M} = \vvv{r}\times\vvv{F}
  \]
  Cuando las fuerzas son coplanares, como en nuestro caso, el sentido de los momentos se
  puede representar mediante un signo.
  En el apéndice~\ref{chapt:ap-momento-fuerza} se repasa brevemente el cálculo del momento de
  una fuerza.

  Supongamos que elegimos el centro $O$ en el punto de contacto de la barra con el
  suelo\footnotemark{}.
  \footnotetext{Elegimos este punto para que el momento de dos fuerzas, $\vvv{N}_1$ y
    $\vvv{f}_r$ sea nulo, reduciendo así el número de términos en la ecuación.}
    
    La suma de momentos alrededor del centro que hemos elegido es nula si la barra está en
    equilibrio. En este ejemplo calculamos los momentos multiplicando el módulo de cada
    fuerza por la distancia desde el centro a la recta de acción a la fuerza.
    \[
      \sum_i \vvv{M}_i = \vvv{0}
    \]
    \[
      \vvv{M}_{N_1} + \vvv{M}_{f_r} + \vvv{M}_P + \vvv{M}_{N_2}  = \vvv{0}
    \]
    \[
      N_1 \cdot 0 + f_r\cdot 0 + mg\cdot \frac{L}{2} \cos\theta - N_2\cdot L\sin\theta = 0
    \]
    %\[
    %  \frac{1}{2}\,mg\cos\theta = N_2\sin\theta
    %\]
    \begin{equation}\label{eq:el-ej-momentos-n2}
      N_2 = \frac{1}{2} mg\cot\theta
    \end{equation}
    Nos queda un sistema de tres ecuaciones, \ref{eq:el-ej-fuerzas-n2},
    \ref{eq:el-ej-fuerzas-n1} y \ref{eq:el-ej-momentos-n2}, con tres incógnitas,
    $N_1$, $N_2$ y $f_r$.

    A continuación, presentamos el resultado de las fuerzas incógnita obtenidas al
    aplicar las leyes de Newton al sistema.
    \begin{align}
      \label{eq:el-newton-n1}
      N_1 &= mg\\
      \label{eq:el-newton-nw}
      N_2 &= \frac{1}{2} mg\cot\theta\\
      \label{eq:el-newton-fr}
      f_r &= \frac{1}{2} mg\cot\theta
    \end{align}

    \item Aplicando el \emph{Principio de D'Alembert}:

      Tenemos dos fuerzas de ligadura, $\vvv{N}_1$ y $\vvv{N}_2$, pues la barra debe estar en
      contacto con la pared y el suelo. La fuerza de rozamiento es una fuerza aplicada,
      realizada por el suelo debido a la fricción que intenta impedir el \emph{deslizamiento}
      de la barra.

      Según la expresión del Principio aplicado a un sistema estático,
      ecuación~(\ref{eq:el-PD-estatico}).
      \[
        \sum_i \vvv{F}_i^{(a)} \cdot \delta\vvv{r}_i = 0
      \]
      obtenemos
      \begin{equation}\label{eq:el-dalembert-barra}
        \vvv{P}\cdot\delta\vvv{r}_p + \vvv{f}_r\cdot\delta\vvv{r}_r = 0
      \end{equation}
      donde $\vvv{r}_p$ es el vector que va desde el origen de coordenadas que pondremos
      en la esquina inferior izquierda, hasta el punto de aplicación del peso, y
      $\vvv{r}_r$ va desde el origen hasta el punto de aplicación de la fuerza de
      rozamiento en el suelo. Ver figura~\ref{fig:el-barra-PD-estatico}.
      \begin{figure}[ht]
        % Escala
        \def\scl{.79}
        \centering
        \begin{subfigure}[b]{.45\textwidth}
          % Barra
          \pgfmathsetmacro{\LBARRA}{5.5}
          \pgfmathsetmacro{\ANGBARRA}{30}
          \pgfmathsetmacro{\RADIOANGBARRA}{2.7}
          \pgfmathsetmacro{\LHORBARRA}{\LBARRA * cos(\ANGBARRA)}
          \pgfmathsetmacro{\LVERBARRA}{\LBARRA * sin(\ANGBARRA)}
          % Suelo
          \pgfmathsetmacro{\LSUELO}{\LHORBARRA + .8}
          \pgfmathsetmacro{\LPARED}{\LVERBARRA + .7}
          % Soporte
          \pgfmathsetmacro{\DISTSOPORTE}{.3}
          \pgfmathsetmacro{\LSOPORTE}{1.1}
          \pgfmathsetmacro{\ANGSOPORTE}{230}
          \pgfmathsetmacro{\NUMTOTHOR}{int(\LSUELO / \DISTSOPORTE)}
          \pgfmathsetmacro{\NUMTOTVER}{int(\LPARED / \DISTSOPORTE)}
          % Peso
          \pgfmathsetmacro{\LPESO}{2.6}
          % Fuerza de rozamiento
          \pgfmathsetmacro{\LFR}{4.0}
          % N1
          \pgfmathsetmacro{\LNSUELO}{\LPESO}
          % N2
          \pgfmathsetmacro{\LNPARED}{\LFR}
          % Fondo
          \pgfmathsetmacro{\BGTOP}{3.8}
          \pgfmathsetmacro{\BGBOTTOM}{-.8}
          \pgfmathsetmacro{\BGLEFT}{-.8}
          \pgfmathsetmacro{\BGRIGHT}{6.0}
          % 
          \centering
          \begin{tikzpicture}[%
            scale=\scl,
            baseline,
            punto fuerza/.style={fill=green!60!black, draw=green!60!black, radius=1.1pt},
            punto/.style={fill=black, draw=black, radius=1.1pt},
            angulo/.style={draw=green!50!black, fill=green!20},
            every node/.style={black,font=\small},
            barra/.style={line width=1.4pt, black!40},
            fuerza/.style={-{Latex[length=8pt,width=4.5pt]}, line width=1.4pt, green!60!black},
            vector posicion/.style={-{Latex[length=8pt,width=4.5pt]}, line width=1.4pt, red!80!black},
            vector/.style={-{Latex[width=5pt]}, line width=1.4pt, black},
            background/.style={%
              line width=\bgborderwidth,
              draw=\bgbordercolor,
              fill=\bgcolor,
            },
            ]
            % COORDENADAS
            \coordinate (O) at (0,0);
            \coordinate (xsuelo) at ($(O) + (\LSUELO, 0)$);
            \coordinate (ypared) at ($(O) + (0,\LPARED)$);
            \coordinate (barra1) at ($(O) + (\LHORBARRA, 0)$);
            \coordinate (barra2) at ($(O) + (0,\LVERBARRA)$);
            \coordinate (peso) at ($(O) + (.5*\LHORBARRA,.5*\LVERBARRA)$);
            % 
            \coordinate (btop) at (0,\BGTOP);
            \coordinate (bbottom) at (0,\BGBOTTOM);
            \coordinate (bleft) at (\BGLEFT,0);
            \coordinate (bright) at (\BGRIGHT,0);
            % DIBUJO
            % Suelo
            \draw (O) -- (xsuelo);
            % Pared
            \draw (O) -- (ypared);
            % Barra
            \draw[barra] (barra1) -- (barra2);
            % Soporte horizontal
            \foreach \xn in {1,2,...,\NUMTOTHOR}
            \draw[black!15, ultra thin] (\DISTSOPORTE * \xn, 0) -- +(\ANGSOPORTE:\LSOPORTE em);
            % Soporte vertical
            \foreach \yn in {1,2,...,\NUMTOTVER}
            \draw[black!15, ultra thin] (0,\DISTSOPORTE * \yn) -- +(\ANGSOPORTE:\LSOPORTE em);
            % r_1
            \draw[vector posicion] (O) -- node[above left=-2pt and -2pt ,red!80!black]
            {$\vvv{r}_p$} (peso);
            % r_2
            \draw[vector posicion] ($(O) +(0,-2.5pt)$) -- node[below,red!80!black]
            {$\vvv{r}_r$} ($(barra1) +(0,-2.5pt)$);
            % vector d
            \draw[vector] (barra1) -- node[above right=-2pt and -2pt] {$\vvv{d}$} (peso);
            % Peso
            \draw[fuerza] (peso) -- node[below left=2pt and 3pt, green!50!black]
            {$P$} +(down:\LPESO em);
            \filldraw[punto fuerza] (peso) circle;
            % fr
            \draw[fuerza] (barra1) -- +(left:\LFR em)
            node[above left=0pt and -8pt,green!50!black] {$f_r$};
            % N1
            \draw[fuerza] (barra1) -- node[above right=0pt and 0pt,green!50!black]
            {$N_1$} +(up:\LNSUELO em);
            \filldraw[punto fuerza] (barra1) circle;
            % N2
            \draw[fuerza] (barra2) --
            node[above right=0pt and 0pt,green!50!black] {$N_2$} +(right:\LNPARED em);
            \filldraw[punto fuerza] (barra2) circle;
            % Origen de coordenadas
            \filldraw[punto] (O) circle;
            \node[below left] at (O) {$O$};
            % YELLOW BACKGROUND
            \begin{scope}[on background layer]
              \node [background, fit= (bleft) (bright) (btop) (bbottom)] {};
            \end{scope}
          \end{tikzpicture}
          \caption{}
          \label{fig:el-barra-PD-estatico}
        \end{subfigure}
        \hspace{2em}
        \begin{subfigure}[b]{.45\textwidth}
          % Barra
          \pgfmathsetmacro{\LBARRA}{5.5}
          \pgfmathsetmacro{\ANGBARRA}{30}
          \pgfmathsetmacro{\RADIOANGBARRA}{2.4}
          \pgfmathsetmacro{\LHORBARRA}{\LBARRA * cos(\ANGBARRA)}
          \pgfmathsetmacro{\LVERBARRA}{\LBARRA * sin(\ANGBARRA)}
          % Suelo
          \pgfmathsetmacro{\LSUELO}{\LHORBARRA + .8}
          \pgfmathsetmacro{\LPARED}{\LVERBARRA + .7}
          % Soporte
          \pgfmathsetmacro{\DISTSOPORTE}{.3}
          \pgfmathsetmacro{\LSOPORTE}{1.1}
          \pgfmathsetmacro{\ANGSOPORTE}{230}
          \pgfmathsetmacro{\NUMTOTHOR}{int(\LSUELO / \DISTSOPORTE)}
          \pgfmathsetmacro{\NUMTOTVER}{int(\LPARED / \DISTSOPORTE)}
          % Peso
          \pgfmathsetmacro{\LPESO}{2.6}
          % Fuerza de rozamiento
          \pgfmathsetmacro{\LFR}{4.0}
          % N1
          \pgfmathsetmacro{\LNSUELO}{\LPESO}
          % N2
          \pgfmathsetmacro{\LNPARED}{\LFR}
          % Fondo
          \pgfmathsetmacro{\BGTOP}{3.8}
          \pgfmathsetmacro{\BGBOTTOM}{-.8}
          \pgfmathsetmacro{\BGLEFT}{-.8}
          \pgfmathsetmacro{\BGRIGHT}{6.0}
          % 
          \centering
          \begin{tikzpicture}[%
            scale=\scl,
            baseline,
            punto fuerza/.style={fill=green!60!black, draw=green!60!black, radius=1.1pt},
            punto/.style={fill=black, draw=black, radius=1.1pt},
            angulo/.style={draw=green!50!black, fill=green!20},
            every node/.style={black,font=\small},
            barra/.style={line width=1.4pt, black!40},
            fuerza/.style={-{Latex[length=8pt,width=4.5pt]}, line width=1.4pt, green!60!black},
            vector posicion/.style={-{Latex[length=8pt,width=4.5pt]}, line width=1.4pt, red!80!black},
            vector/.style={-{Latex[length=8pt,width=4.5pt]}, line width=1.4pt, black},
            background/.style={%
              line width=\bgborderwidth,
              draw=\bgbordercolor,
              fill=\bgcolor,
            },
            ]
            % COORDENADAS
            \coordinate (O) at (0,0);
            \coordinate (xsuelo) at ($(O) + (\LSUELO, 0)$);
            \coordinate (ypared) at ($(O) + (0,\LPARED)$);
            \coordinate (barra1) at ($(O) + (\LHORBARRA, 0)$);
            \coordinate (barra2) at ($(O) + (0,\LVERBARRA)$);
            \coordinate (peso) at ($(O) + (.5*\LHORBARRA,.5*\LVERBARRA)$);
            % Perpendicular al suelo
            \path[name path=vertical] (peso) -- +(down:1.6);
            \path[name path=horizontal] (O) -- (barra1);
            \path[name intersections={of=vertical and horizontal,by=M}];
            % 
            \coordinate (btop) at (0,\BGTOP);
            \coordinate (bbottom) at (0,\BGBOTTOM);
            \coordinate (bleft) at (\BGLEFT,0);
            \coordinate (bright) at (\BGRIGHT,0);
            % DIBUJO
            % Ángulo
            \filldraw[angulo,-{Latex[round]}]
            (barra1) -- +(left:\RADIOANGBARRA em)
            arc [start angle=180, end angle=180-\ANGBARRA, radius=\RADIOANGBARRA em] -- cycle;
            \node[above left=-2.77pt and 8.5pt] at (barra1) {\footnotesize $\theta$};
            % Suelo
            \draw (O) -- (xsuelo);
            % Pared
            \draw (O) -- (ypared);
            % Barra
            \draw[barra] (barra1) -- (barra2);
            % Soporte horizontal
            \foreach \xn in {1,2,...,\NUMTOTHOR}
            \draw[black!15, ultra thin] (\DISTSOPORTE * \xn, 0) -- +(\ANGSOPORTE:\LSOPORTE em);
            % Soporte vertical
            \foreach \yn in {1,2,...,\NUMTOTVER}
            \draw[black!15, ultra thin] (0,\DISTSOPORTE * \yn) -- +(\ANGSOPORTE:\LSOPORTE em);
            % r_1
            % \draw[] (O) -- node[above left=-2pt and -2pt ,red!80!black] {$\vvv{r_1}$} (peso);
            % r_2
            % \draw[red!80!black] (O) -- node[below,red!80!black] {$a$} (barra1);
            % Hipotenusa
            \draw[-{Latex[length=8pt, width=4.5pt]}, shorten >=-3pt, black, line width=1.8pt]
            (barra1) -- node[above right=-2pt and -2pt] {$|\vvv{d}|=L/2$} (peso);
            % Cateto suelo
            \draw[{Latex[length=8pt, width=4.5pt]}-, shorten >=3pt, red!80!black, line width=1.2pt]
            (M) -- node[below] {$x$} (barra1);
            % Cateto vertical
            \draw[{Latex[length=8pt, width=4.5pt]}-, shorten >=0pt, red!80!black, line width=1.2pt]
            (peso) -- node[left] {$y$} (M);
            % YELLOW BACKGROUND
            \begin{scope}[on background layer]
              \node [background, fit= (bleft) (bright) (btop) (bbottom)] {};
            \end{scope}
          \end{tikzpicture}
          \caption{}
          \label{fig:el-triangulo-rectangulo}
        \end{subfigure}
        \caption{En la figura (a) se especifica el origen de coordenadas $O$ elegido. Los vectores
          posición del peso $\vvv{r}_p$ y de la fuerza de rozamiento $\vvv{r}_r$ se resaltan en rojo.
          Nótese que para que no se solape la fuerza de rozamiento con $\vvv{r}_r$, se ha desplazado
          ligeramente hacia abajo este último vector. El vector $\vvv{d}$ sobre la barra relaciona
          $\vvv{r}_p$ con $\vvv{r}_r$, de manera que $\vvv{r}_p = \vvv{r}_r + \vvv{d}$.
          En la figura (b) tenemos el triángulo rectángulo que nos permitirá calcular las componentes
          cartesianas del vector $\vvv{d}$.}
      \end{figure}
      
      En dicha figura hemos relacionado los vectores $\vvv{r}_p$ y $\vvv{r}_r$ a través de un vector
      $\vvv{d}$ que está sobre la barra y se dirige desde el punto de aplicación de la fuerza de
      rozamiento hasta el del peso, esto es, su módulo es $|\vvv{d}|=L/2$.
      Según esta definición de $\vvv{d}$, la relación entre $\vvv{r}_p$ y $\vvv{r}_r$ es
      \begin{equation}\label{eq:el-rp}
        \vvv{r}_p = \vvv{r}_r + \vvv{d}
      \end{equation}

      Primero se calculan los catetos $x$ e $y$ con ayuda del triángulo rectángulo de la
      figura~\ref{fig:el-triangulo-rectangulo}, donde
      \begin{align*}
        x &= \frac{L}{2}\cos\theta\\
        y &= \frac{L}{2}\sin\theta
      \end{align*}
     
      Ahora se obtienen $\vvv{r}_r$ y $\vvv{d}$
      \begin{align}\label{eq:el-rr}
        \vvv{r}_r &= (2x,0) = (L\cos\theta,0)\\
        \label{eq:el-d}
        \vvv{d} &= (-x,y) = \left(-\frac{L}{2}\cos\theta, \frac{L}{2}\sin\theta\right)
      \end{align}

      Calculamos $\vvv{r}_p$ utilizando la relación~(\ref{eq:el-rp}) con ayuda de los
      valores~\ref{eq:el-rr} y~\ref{eq:el-d}
      \begin{equation}\label{eq:el-rp-dos}
        \vvv{r}_p
        = \vvv{r}_r + \vvv{d}
        = \left(L\cos\theta,0\right) + \left(-\frac{L}{2}\cos\theta, \frac{L}{2}\sin\theta\right)
        = \left(\frac{L}{2}\cos\theta, \frac{L}{2}\sin\theta\right)
      \end{equation}

      La variación virtual de $\vvv{r}_p$ es
      \begin{align*}
        \delta \vvv{r}_p
        &= \frac{\partial\vvv{r}_p}{\partial L}\,\cancelout{\delta L}
          + \frac{\partial\vvv{r}_p}{\partial\theta}\,\delta\theta
           = \frac{\partial\vvv{r}_p}{\partial\theta}\,\delta\theta
          = \left(-\frac{L}{2}\sin\theta, \frac{L}{2}\cos\theta\right)\,\delta\theta
        \\
        \delta \vvv{r}_r
        &= \frac{\partial\vvv{r}_r}{\partial L}\,\cancelout{\delta L}
          + \frac{\partial\vvv{r}_r}{\partial\theta}\,\delta\theta
          = \frac{\partial\vvv{r}_r}{\partial\theta}\,\delta\theta
          = (-L\sin\theta, 0)\,\delta\theta
      \end{align*}
      donde se ha tenido en cuenta que $\delta L = 0$ porque al desplazarse ligeramente la barra
      respetando las ligaduras, no cambia su longitud.

      Con los valores calculados, aplicamos el principio de D'Alembert a la
      barra, ecuación (\ref{eq:el-dalembert-barra}).
      \[
        (0,-mg) \cdot \frac{L}{2} (-\sin\theta, \cos\theta)\,\delta\theta
        + (-f_r,0)\cdot (-L\sin\theta, 0)\,\delta\theta = 0
      \]
      Se desarrolla la expresión, obteniéndose
      \[
        \left(-\frac{mg}{2}\cos\theta + f_r\sin\theta\right)\,\delta\theta = 0
      \]
      Como $\delta\theta$ no puede ser cero en un desplazamiento virtual, nos queda que
      \[
        -\frac{mg}{2}\cos\theta + f_r\sin\theta
      \]
      \[
        f_r\sin\theta  = \frac{mg}{2}\cos\theta
      \]
      \[
        f_r = \frac{mg}{2}\cot\theta
      \]
      que coincide con el resultado~\ref{eq:el-newton-fr}, que se obtuvo aplicando la segunda
      condición de equilibrio $\sum_i \vvv{M}_i = 0$, aunque sigue siendo necesario aplicar
      $\sum_i \vvv{F}_i = 0$ para obtener las fuerzas de ligadura, $\vvv{N}_1$ y $\vvv{N}_2$.
    \end{itemize}

    \subsection{Principio de D'Alembert dinámico}
    Las ecuaciones de Euler-Lagrange se obtuvieron por primera vez aplicando el principio de D'Alembert
    dinámico, no mediante un principio variacional.

    La versión dinámica del principio de D'Alembert es \begin{equation}\label{eq:el-PD-dinamico}
    \sum_{i=1}^N \left(\dot{\vvv{p}}_i - \vvv{F}_i^{(a)}\right)\cdot \delta \vvv{r}_i = 0
    \end{equation}
    donde $\vvv{\dot{p}_i} = d\vvv{p}_i/dt$, con $\vvv{p}_i = m_i\vvv{v}_i$ siendo el momento lineal
    de la partícula $i$, y $\vvv{F}_i$ es la fuerza aplicada a dicha partícula.
    Nótese además, que si la masa de la partícula es constante
    \[
      \vvv{\dot{p}}_i
      = \frac{d\vvv{p}_i}{dt}
      = \frac{d(m_i\vvv{v}_i)}{dt}
      = m_i\frac{d\vvv{v}_i}{dt} + \cancelout{\dfrac{dm_i}{dt}}\,\vvv{v}_i
      = m_i \vvv{a}_i
    \]
    donde $\vvv{a}_i$ es la aceleración de la partícula $i$.
    En este supuesto, la ecuación~(\ref{eq:el-PD-dinamico}) se puede reescribir como
    \[
      \sum_{i=1}^N \left(m_i\vvv{a}_i - \vvv{F}_i^{(a)}\right)\cdot \delta \vvv{r}_i = 0
    \]
    

    \subsubsection{Ejercicio}
    Consideremos una partícula de masa $m=\SI{100}{\gram}$ que desliza sin rozamiento desde lo alto y a
    lo largo de una barra inclinada según se esquematiza en la figura~\ref{fig:el-barra2-geometria}.
    Las fuerzas que actúan sobre ella se representan en la figura~\ref{fig:el-barra2-fuerzas}.
    La fuerza aplicada es el peso, $\vvv{P}$, cuyo módulo es $P=mg$. Debido a las características del problema,
    tenemos una fuerza de ligadura sobre la partícula, que es la normal $\vvv{N}$, perpendicular a la barra.

    \begin{figure}[ht]
      % Escala
      \def\scl{.75}
      \centering
      \begin{subfigure}[b]{.43\linewidth}
        % Barra
        \pgfmathsetmacro{\LBARRA}{5.5}
        \pgfmathsetmacro{\ANGBARRA}{60}
        \pgfmathsetmacro{\RADIOANGBARRA}{2.6}
        \pgfmathsetmacro{\LHORBARRA}{\LBARRA * cos(\ANGBARRA)}
        \pgfmathsetmacro{\LVERBARRA}{\LBARRA * sin(\ANGBARRA)}
        % P
        % Partícula
        \pgfmathsetmacro{\LPART}{.70*\LBARRA}
        % Suelo
        \pgfmathsetmacro{\LSUELO}{\LHORBARRA + .8}
        \pgfmathsetmacro{\LPARED}{\LVERBARRA + .7}
        % Soporte
        \pgfmathsetmacro{\DISTSOPORTE}{.3}
        \pgfmathsetmacro{\LSOPORTE}{1.1}
        \pgfmathsetmacro{\ANGSOPORTE}{230}
        \pgfmathsetmacro{\NUMTOTHOR}{int(\LSUELO / \DISTSOPORTE)}
        \pgfmathsetmacro{\NUMTOTVER}{int(\LPARED / \DISTSOPORTE)}
        % Fondo
        \pgfmathsetmacro{\BGTOP}{5.9}
        \pgfmathsetmacro{\BGBOTTOM}{-1.6}
        \pgfmathsetmacro{\BGLEFT}{-1.9}
        \pgfmathsetmacro{\BGRIGHT}{4.3}
        % 
        \centering
        \begin{tikzpicture}[%
          scale=\scl,
          baseline,
          punto barra/.style={fill=black, draw=black, radius=.5pt},
          angulo/.style={draw=green!50!black, fill=green!20},
          every node/.style={black,font=\small},
          barra/.style={black!50},
          fuerza/.style={-{Latex[length=8pt,width=4.5pt]}, line width=1pt},
          particula/.style={fill=red, draw=black, radius=2.5pt},
          limite/.style={black!50},
          soporte/.style={black!25, ultra thin},
          linea auxiliar/.style={lightgray, ultra thin},
          background/.style={%
            line width=\bgborderwidth,
            draw=\bgbordercolor,
            fill=\bgcolor,
          },
          ]
          % COORDENADAS
          \coordinate (O) at (0,0);
          \coordinate (xsuelo) at ($(O) + (\LSUELO, 0)$);
          \coordinate (ypared) at ($(O) + (0,\LPARED)$);
          \coordinate (barra1) at ($(O) + (\LHORBARRA, 0)$);
          \coordinate (barra2) at ($(O) + (0,\LVERBARRA)$);
          \path (barra1) -- +(180-\ANGBARRA:\LPART) coordinate (part);
          % 
          \coordinate (btop) at (0,\BGTOP);
          \coordinate (bbottom) at (0,\BGBOTTOM);
          \coordinate (bleft) at (\BGLEFT,0);
          \coordinate (bright) at (\BGRIGHT,0);
          % DIBUJO
          % Medidas
          % pared
          \draw[linea auxiliar] (O) -- +(left:1.3) coordinate (LO);
          \draw[linea auxiliar] (barra2) -- +(left:1.3) coordinate (Lbarra2);
          \path (O) -- node[left=.25] {$\SI{2}{\metre}$} (barra2);
          \draw[{Straight Barb}-, lightgray] ($(LO)!.32!(O)$) -- +(up:2.1);
          \draw[{Straight Barb}-, lightgray] ($(Lbarra2)!.32!(barra2)$) -- +(down:2.1);
          % suelo
          \draw[linea auxiliar] (O) -- +(down:1.1) coordinate (BO);
          \draw[linea auxiliar] (barra1) -- +(down:1.1) coordinate (Bbarra1);
          \path (O) -- node[below=.3] {$\SI{1}{\metre}$} (barra1);
          \draw[{Straight Barb}-, lightgray] ($(BO)!.31!(O)$) -- +(right:1.0);
          \draw[{Straight Barb}-, lightgray] ($(Bbarra1)!.31!(barra1)$) -- +(left:1.0);
          % Ángulo
          \filldraw[angulo,-{Latex[round]}]
          (barra1) -- +(left:\RADIOANGBARRA em)
          arc [start angle=180, end angle=180-\ANGBARRA, radius=\RADIOANGBARRA em] -- cycle;
          \node[above left=0pt and 5.0pt] at (barra1) {$\theta$};
          % Suelo
          \draw[limite] (O) -- (xsuelo);
          % Pared
          \draw[limite] (O) -- (ypared);
          % Barra
          \draw[barra] (barra1) -- node[above right=-2pt and -2pt]
          {$L=\sqrt{5}\,\si{\metre}$} (barra2);
          % Extremos de la barra
          \filldraw[punto barra] (barra1) circle;
          \filldraw[punto barra] (barra2) circle;
          % Partícula
          \filldraw[particula] (part) circle;
          \node[above right] at (part) {$m$};
          % Soporte horizontal
          \foreach \xn in {1,2,...,\NUMTOTHOR}
          \draw[soporte] (\DISTSOPORTE * \xn, 0) -- +(\ANGSOPORTE:\LSOPORTE em);
          % Soporte vertical
          \foreach \yn in {1,2,...,\NUMTOTVER}
          \draw[soporte] (0,\DISTSOPORTE * \yn) -- +(\ANGSOPORTE:\LSOPORTE em);
          % YELLOW BACKGROUND
          \begin{scope}[on background layer]
            \node [background, fit= (bleft) (bright) (btop) (bbottom)] {};
          \end{scope}
        \end{tikzpicture}
        \caption{}
        \label{fig:el-barra2-geometria}
      \end{subfigure}
      \hspace{2em}
      \begin{subfigure}[b]{.43\linewidth}
        % Barra
        \pgfmathsetmacro{\LBARRA}{5.5}
        \pgfmathsetmacro{\ANGBARRA}{60}
        \pgfmathsetmacro{\RADIOANGBARRA}{1.05}
        \pgfmathsetmacro{\LHORBARRA}{\LBARRA * cos(\ANGBARRA)}
        \pgfmathsetmacro{\LVERBARRA}{\LBARRA * sin(\ANGBARRA)}
        % Posición de la partícula en la barra contanto desde el suelo
        \pgfmathsetmacro{\LPART}{.5*\LBARRA}
        % Suelo
        \pgfmathsetmacro{\LSUELO}{\LHORBARRA + .8}
        \pgfmathsetmacro{\LPARED}{\LVERBARRA + .7}
        % Material de soporte suelo y pared
        \pgfmathsetmacro{\DISTSOPORTE}{.3}
        \pgfmathsetmacro{\LSOPORTE}{1.1}
        \pgfmathsetmacro{\ANGSOPORTE}{230}
        \pgfmathsetmacro{\NUMTOTHOR}{int(\LSUELO / \DISTSOPORTE)}
        \pgfmathsetmacro{\NUMTOTVER}{int(\LPARED / \DISTSOPORTE)}
        % Peso
        % Módulo del peso
        \pgfmathsetmacro{\LPESO}{2.0}
        % Componente del peso perpendicular a la barra
        \pgfmathsetmacro{\PESOPERP}{\LPESO * cos(\ANGBARRA)}
        % Normal
        \pgfmathsetmacro{\LNORMAL}{\PESOPERP}
        % Fondo
        \pgfmathsetmacro{\BGTOP}{5.9}
        \pgfmathsetmacro{\BGBOTTOM}{-1.6}
        \pgfmathsetmacro{\BGLEFT}{-1.9}
        \pgfmathsetmacro{\BGRIGHT}{4.3}
        % 
        \centering
        \begin{tikzpicture}[%
          scale=\scl,
          baseline,
          punto fuerza/.style={fill=green!60!black, draw=green!60!black, radius=1.1pt},
          punto barra/.style={fill=black, draw=black, radius=.5pt},
          angulo/.style={draw=green!50!black, fill=green!20},
          every node/.style={black,font=\small},
          barra/.style={black!50},
          fuerza/.style={-{Latex[length=8pt,width=4.5pt]}, shorten >=-2pt, line width=1.4pt, green!60!black},
          particula/.style={fill=red, draw=black, radius=2.5pt},
          limite/.style={black!50},
          soporte/.style={black!25, ultra thin},
          linea auxiliar/.style={lightgray, ultra thin},
          background/.style={%
            line width=\bgborderwidth,
            draw=\bgbordercolor,
            fill=\bgcolor,
          },
          ]
          % COORDENADAS
          \coordinate (O) at (0,0);
          \coordinate (xsuelo) at ($(O) + (\LSUELO, 0)$);
          \coordinate (ypared) at ($(O) + (0,\LPARED)$);
          \coordinate (barra1) at ($(O) + (\LHORBARRA, 0)$);
          \coordinate (barra2) at ($(O) + (0,\LVERBARRA)$);
          \coordinate (peso) at (part);
          \path (barra1) -- +(180-\ANGBARRA:\LPART) coordinate (part);
          \path (part) -- +(down:\LPESO) coordinate (peso);
          \path (part) -- +(90-\ANGBARRA:\LNORMAL) coordinate (normal);
          % 
          \coordinate (btop) at (0,\BGTOP);
          \coordinate (bbottom) at (0,\BGBOTTOM);
          \coordinate (bleft) at (\BGLEFT,0);
          \coordinate (bright) at (\BGRIGHT,0);
          % DIBUJO
          % Medidas
          % pared
          \draw[linea auxiliar] (O) -- +(left:1.3) coordinate (LO);
          \draw[linea auxiliar] (barra2) -- +(left:1.3) coordinate (Lbarra2);
          \path (O) -- node[left=.25] {$\SI{2}{\metre}$} (barra2);
          \draw[{Straight Barb}-, lightgray] ($(LO)!.32!(O)$) -- +(up:2.1);
          \draw[{Straight Barb}-, lightgray] ($(Lbarra2)!.32!(barra2)$) -- +(down:2.1);
          % suelo
          \draw[lightgray, ultra thin] (O) -- +(down:1.1) coordinate (BO);
          \draw[lightgray, ultra thin] (barra1) -- +(down:1.1) coordinate (Bbarra1);
          \path (O) -- node[below=.3] {$\SI{1}{\metre}$} (barra1);
          \draw[{Straight Barb}-, lightgray] ($(BO)!.31!(O)$) -- +(right:1.0);
          \draw[{Straight Barb}-, lightgray] ($(Bbarra1)!.31!(barra1)$) -- +(left:1.0);
          % Ángulo en la barra
          \filldraw[angulo,-{Latex[round]}]
          (barra1) -- +(left:\RADIOANGBARRA em)
          arc [start angle=180, end angle=180-\ANGBARRA, radius=\RADIOANGBARRA em] -- cycle;
          % Suelo
          \draw[limite] (O) -- (xsuelo);
          % Pared
          \draw[limite] (O) -- (ypared);
          % Barra
          \draw[barra] (barra1) -- (barra2);
          % Extremos de la barra
          \filldraw[punto barra] (barra1) circle;
          \filldraw[punto barra] (barra2) circle;
          % Soporte horizontal
          \foreach \xn in {1,2,...,\NUMTOTHOR}
          \draw[soporte] (\DISTSOPORTE * \xn, 0) -- +(\ANGSOPORTE:\LSOPORTE em);
          % Soporte vertical
          \foreach \yn in {1,2,...,\NUMTOTVER}
          \draw[soporte] (0,\DISTSOPORTE * \yn) -- +(\ANGSOPORTE:\LSOPORTE em);
          % Peso
          \draw[fuerza] (part) -- (peso);
          \node[below left=-5pt and 0pt] at (peso) {$\vvv{P}$};
          % Normal
          \draw[fuerza] (part) -- (normal);
          \node[above] at (normal) {$\vvv{N}$};
          % Partícula
          \filldraw[particula] (part) circle;
          % YELLOW BACKGROUND
          \begin{scope}[on background layer]
            \node [background, fit= (bleft) (bright) (btop) (bbottom)] {};
          \end{scope}
        \end{tikzpicture}
        \caption{}
        \label{fig:el-barra2-fuerzas}
      \end{subfigure}
      \caption{Una partícula de masa $m$ desliza sin rozamiento a lo largo de una barra de longitud
        $L=\sqrt{5}\,\si{\metre}$ que forma un ángulo con la horizontal. En la imagen $(a)$ se
        considera la geometría del sistema, mientras que en la $(b)$ se representan las fuerzas que
        actúan sobre la partícula.}
    \end{figure}
Primero calculamos el seno y el coseno del ángulo $\theta$ que forma la barra con el suelo
\begin{align}
  \label{eq:el-barra2-sin}
  \sin\theta &= \frac{2\,\cancelout{\si{\metre}}}{\sqrt{5}\,\cancelout{\si{\metre}}}
               = \frac{2}{\sqrt{5}}\\
  \label{eq:el-barra2-cos}
  \cos\theta &= \frac{1\,\cancelout{\si{\metre}}}{\sqrt{5}\,\cancelout{\si{\metre}}}
               = \frac{1}{\sqrt{5}}
\end{align}

Vamos a resolver el problema de dos formas:
\begin{itemize}
\item Aplicando las \emph{leyes de Newton}:

  Para este caso, utilizaremos un sistema de coordenadas con origen en $O'$, situado en el extremo
  superior de la barra, donde el eje $x'$ es paralelo a la barra y con sentido hacia abajo,
  mientras que el eje $y'$ es perpendicular a la barra y se dirige hacia arriba.
  Hacemos esto porque sabemos de antemano que la ligadura $\vvv{N}$ va a impedir que la partícula
  tenga aceleración perpendicular a la barra $a_{y'} = 0$. Cuando se calculen las componentes en
  sistema de coordenadas, pasaremos los resultados al sistema de coordenadas con centro en $O$
  en la unión de la pared con el suelo, con ejes $x$ e $y$  horizontal y vertical, respectivamente.
  Ver figura~\ref{fig:el-barra2-fuerzas-ejes}.
  
  Además, en la figura~\ref{fig:el-barra2-fuerzas-par-perp}, se ha descompuesto el peso en sus
  componentes, paralela y perpendicular a la barra, $\vvv{P}_{x'}$ y $\vvv{P}_{y'}$,
  respectivamente.
    
  \begin{figure}[ht]
    % Escala
    \def\scl{.75}
    % 
    \centering
    \begin{subfigure}[b]{.35\textwidth}
    % Barra
    \pgfmathsetmacro{\LBARRA}{5.5}
    \pgfmathsetmacro{\ANGBARRA}{60}
    \pgfmathsetmacro{\RADIOANGBARRA}{1.05}
    \pgfmathsetmacro{\LHORBARRA}{\LBARRA * cos(\ANGBARRA)}
    \pgfmathsetmacro{\LVERBARRA}{\LBARRA * sin(\ANGBARRA)}
    % Posición de la partícula en la barra contanto desde el suelo
    \pgfmathsetmacro{\LPART}{.5*\LBARRA}
    % Suelo
    \pgfmathsetmacro{\LSUELO}{\LHORBARRA + .8}
    \pgfmathsetmacro{\LPARED}{\LVERBARRA + .7}
    % Material de soporte suelo y pared
    \pgfmathsetmacro{\DISTSOPORTE}{.3}
    \pgfmathsetmacro{\LSOPORTE}{1.1}
    \pgfmathsetmacro{\ANGSOPORTE}{230}
    \pgfmathsetmacro{\NUMTOTHOR}{int(\LSUELO / \DISTSOPORTE)}
    \pgfmathsetmacro{\NUMTOTVER}{int(\LPARED / \DISTSOPORTE)}
    % Peso
    % Módulo del peso
    \pgfmathsetmacro{\LPESO}{2.0}
    % Componente del peso paralela a la barra
    \pgfmathsetmacro{\PESOPAR}{\LPESO * sin(\ANGBARRA)}
    % Componente del peso perpendicular a la barra
    \pgfmathsetmacro{\PESOPERP}{\LPESO * cos(\ANGBARRA)}
    \pgfmathsetmacro{\PESOPERPLINAUX}{1.35*\PESOPERP}
    % Líneas auxiliares
    % Normal
    \pgfmathsetmacro{\LNORMAL}{\PESOPERP}
    % Ejes x' e y'
    \pgfmathsetmacro{\LXPRIMA}{.9}
    \pgfmathsetmacro{\LYPRIMA}{.9}
    % Ejes x e y
    \pgfmathsetmacro{\LX}{.9}
    \pgfmathsetmacro{\LY}{.9}
    %
    \pgfmathsetmacro{\BGTOP}{6.2}
    \pgfmathsetmacro{\BGBOTTOM}{-1.0}
    \pgfmathsetmacro{\BGLEFT}{-1.3}
    \pgfmathsetmacro{\BGRIGHT}{4.4}
    %
    \centering
      \begin{tikzpicture}[%
        scale=\scl,
        baseline,
        punto fuerza/.style={fill=green!60!black, draw=green!60!black, radius=1.1pt},
        punto barra/.style={fill=black, draw=black, radius=.5pt},
        origen eje/.style={fill=black, draw=black, radius=1pt},
        angulo/.style={draw=green!50!black, fill=green!20},
        every node/.style={black,font=\small},
        barra/.style={black!50},
        fuerza/.style={-{Latex[length=6pt,width=4pt]}, shorten >=-2pt, line width=1.0pt, green!60!black},
        componente fuerza/.style={-{Latex[length=8pt,width=4.5pt]}, shorten >=-2pt, line width=1.4pt, black!60},
        particula/.style={fill=red, draw=black, radius=2.5pt},
        eje/.style={-{>}, black, line width=.8pt},
        limite/.style={black!50},
        soporte/.style={black!25, ultra thin},
        linea auxiliar/.style={lightgray, ultra thin},
        background/.style={%
          line width=\bgborderwidth,
          draw=\bgbordercolor,
          fill=\bgcolor,
        },
        ]
        % COORDENADAS
        \coordinate (O) at (0,0);
        \coordinate (xsuelo) at ($(O) + (\LSUELO, 0)$);
        \coordinate (ypared) at ($(O) + (0,\LPARED)$);
        \coordinate (barra1) at ($(O) + (\LHORBARRA, 0)$);
        \coordinate (barra2) at ($(O) + (0,\LVERBARRA)$);
        \coordinate (O') at (barra2);
        \coordinate (peso) at (part);
        \path (barra1) -- +(180-\ANGBARRA:\LPART) coordinate (part);
        \path (part) -- +(down:\LPESO) coordinate (peso);
        \path (part) -- +(360-\ANGBARRA:\PESOPAR) coordinate (pesopar);
        \path (part) -- +(270-\ANGBARRA:\PESOPERP) coordinate (pesoperp);
        \path (part) -- +(90-\ANGBARRA:\LNORMAL) coordinate (normal);
        \path (barra2) -- +(90-\ANGBARRA:\LYPRIMA) coordinate (yprima);
        \path (barra2) -- +(360-\ANGBARRA:\LXPRIMA) coordinate (xprima);
        % 
        \coordinate (btop) at (0,\BGTOP);
        \coordinate (bbottom) at (0,\BGBOTTOM);
        \coordinate (bleft) at (\BGLEFT,0);
        \coordinate (bright) at (\BGRIGHT,0);
        % DIBUJO
        % Ángulo en la barra
        % \filldraw[angulo,-{Latex[round]}]
        % (barra1) -- +(left:\RADIOANGBARRA em)
        % arc [start angle=180, end angle=180-\ANGBARRA, radius=\RADIOANGBARRA em] -- cycle;
        % \node[above left=0pt and 2.9pt] at (barra1) {\footnotesize $\theta$};
        % Ángulo en el peso
        % \filldraw[angulo,-{Latex[round]}]
        % (part) -- +(270-\ANGBARRA:\RADIOANGBARRA em)
        % arc [start angle=270-\ANGBARRA, end angle=270, radius=\RADIOANGBARRA em] -- cycle;
        % Suelo
        \draw[limite] (O) -- (xsuelo);
        % Pared
        \draw[limite] (O) -- (ypared);
        % Barra
        \draw[barra] (barra1) -- (barra2);
        % Extremos de la barra
        \filldraw[punto barra] (barra1) circle;
        \filldraw[punto barra] (barra2) circle;
        % Partícula
        \filldraw[fill=red,draw=black] (part) circle[radius=2pt];
        % Soporte horizontal
        \foreach \xn in {1,2,...,\NUMTOTHOR}
        \draw[soporte] (\DISTSOPORTE * \xn, 0) -- +(\ANGSOPORTE:\LSOPORTE em);
        % Soporte vertical
        \foreach \yn in {1,2,...,\NUMTOTVER}
        \draw[soporte] (0,\DISTSOPORTE * \yn) -- +(\ANGSOPORTE:\LSOPORTE em);
        % Ejes x e y
        \draw[eje] (O) -- +(right:\LX) node[below=2pt] {$x$};
        \draw[eje] (O) -- +(up:\LY) node[left=2pt] {$y$};
        \filldraw[origen eje] (O) circle;
        \node[below left] at (O) {$O$};
        % Ejes x' e y'
        \draw[eje] (barra2) -- (xprima);
        \node[below left=-1pt and -2pt] at (xprima) {$x'$};
        \draw[eje] (barra2) -- (yprima);
        \node[above left=0pt and 0pt] at (yprima) {$y'$};
        \filldraw[origen eje] (O') circle;
        \node[left=2pt] at (O') {$O'$};        
        % Peso
        \draw[fuerza] (part) -- (peso);
        \node[above left=0pt and 2pt, green!60!black] at (peso) {$\vvv{P}$};
        % Normal
        \draw[fuerza] (part) -- (normal);
        \node[above left=0pt and 0pt, green!60!black] at (normal) {$\vvv{N}$};
        % Partícula
        \filldraw[particula] (part) circle;
        % YELLOW BACKGROUND
        \begin{scope}[on background layer]
          \node [background, fit= (bleft) (bright) (btop) (bbottom)] {};
        \end{scope}
      \end{tikzpicture}
      \caption{}
      \label{fig:el-barra2-fuerzas-ejes}
    \end{subfigure}
    \hspace{2em}
    \begin{subfigure}[b]{.35\textwidth}
      % Barra
      \pgfmathsetmacro{\LBARRA}{5.5}
      \pgfmathsetmacro{\ANGBARRA}{60}
      \pgfmathsetmacro{\RADIOANGBARRA}{1.05}
      \pgfmathsetmacro{\LHORBARRA}{\LBARRA * cos(\ANGBARRA)}
      \pgfmathsetmacro{\LVERBARRA}{\LBARRA * sin(\ANGBARRA)}
      % Posición de la partícula en la barra contanto desde el suelo
      \pgfmathsetmacro{\LPART}{.5*\LBARRA}
      % Suelo
      \pgfmathsetmacro{\LSUELO}{\LHORBARRA + .8}
      \pgfmathsetmacro{\LPARED}{\LVERBARRA + .7}
      % Material de soporte suelo y pared
      \pgfmathsetmacro{\DISTSOPORTE}{.3}
      \pgfmathsetmacro{\LSOPORTE}{1.1}
      \pgfmathsetmacro{\ANGSOPORTE}{230}
      \pgfmathsetmacro{\NUMTOTHOR}{int(\LSUELO / \DISTSOPORTE)}
      \pgfmathsetmacro{\NUMTOTVER}{int(\LPARED / \DISTSOPORTE)}
      % Peso
      % Módulo del peso
      \pgfmathsetmacro{\LPESO}{2.0}
      % Componente del peso paralela a la barra
      \pgfmathsetmacro{\PESOPAR}{\LPESO * sin(\ANGBARRA)}
      % Componente del peso perpendicular a la barra
      \pgfmathsetmacro{\PESOPERP}{\LPESO * cos(\ANGBARRA)}
      \pgfmathsetmacro{\PESOPERPLINAUX}{1.35*\PESOPERP}
      % Líneas auxiliares
      % Normal
      \pgfmathsetmacro{\LNORMAL}{\PESOPERP}
      % 
      \pgfmathsetmacro{\BGTOP}{6.2}
      \pgfmathsetmacro{\BGBOTTOM}{-1.0}
      \pgfmathsetmacro{\BGLEFT}{-1.0}
      \pgfmathsetmacro{\BGRIGHT}{4.4}
      % 
      \centering
      \begin{tikzpicture}[%
        scale=\scl,
        baseline,
        punto fuerza/.style={fill=green!60!black, draw=green!60!black, radius=1.1pt},
        punto barra/.style={fill=black, draw=black, radius=.5pt},
        angulo/.style={draw=green!50!black, fill=green!20},
        every node/.style={black,font=\small},
        barra/.style={black!50},
        fuerza/.style={-{Latex[length=8pt,width=4.5pt]}, shorten >=-2pt, line width=1.4pt, green!60!black},
        componente fuerza/.style={-{Latex[length=8pt,width=4.5pt]}, shorten >=-2pt, line width=1.4pt, black!60},
        particula/.style={fill=red, draw=black, radius=2.5pt},
        limite/.style={black!50},
        soporte/.style={black!25, ultra thin},
        linea auxiliar/.style={lightgray, ultra thin},
        background/.style={%
          line width=\bgborderwidth,
          draw=\bgbordercolor,
          fill=\bgcolor,
        },
        ]
        % COORDENADAS
        \coordinate (O) at (0,0);
        \coordinate (xsuelo) at ($(O) + (\LSUELO, 0)$);
        \coordinate (ypared) at ($(O) + (0,\LPARED)$);
        \coordinate (barra1) at ($(O) + (\LHORBARRA, 0)$);
        \coordinate (barra2) at ($(O) + (0,\LVERBARRA)$);
        \coordinate (peso) at (part);
        \path (barra1) -- +(180-\ANGBARRA:\LPART) coordinate (part);
        \path (part) -- +(down:\LPESO) coordinate (peso);
        \path (part) -- +(360-\ANGBARRA:\PESOPAR) coordinate (pesopar);
        \path (part) -- +(270-\ANGBARRA:\PESOPERP) coordinate (pesoperp);
        \path (part) -- +(90-\ANGBARRA:\LNORMAL) coordinate (normal);
        % 
        \coordinate (btop) at (0,\BGTOP);
        \coordinate (bbottom) at (0,\BGBOTTOM);
        \coordinate (bleft) at (\BGLEFT,0);
        \coordinate (bright) at (\BGRIGHT,0);
        % DIBUJO
        % Ángulo en la barra
        \filldraw[angulo,-{Latex[round]}]
        (barra1) -- +(left:\RADIOANGBARRA em)
        arc [start angle=180, end angle=180-\ANGBARRA, radius=\RADIOANGBARRA em] -- cycle;
        % \node[above left=0pt and 2.9pt] at (barra1) {\footnotesize $\theta$};
        % Ángulo en el peso
        \filldraw[angulo,-{Latex[round]}]
        (part) -- +(270-\ANGBARRA:\RADIOANGBARRA em)
        arc [start angle=270-\ANGBARRA, end angle=270, radius=\RADIOANGBARRA em] -- cycle;
        % Suelo
        \draw[limite] (O) -- (xsuelo);
        % Pared
        \draw[limite] (O) -- (ypared);
        % Barra
        \draw[barra] (barra1) -- (barra2);
        % Extremos de la barra
        \filldraw[punto barra] (barra1) circle;
        \filldraw[punto barra] (barra2) circle;
        % Partícula
        \filldraw[fill=red,draw=black] (part) circle[radius=2pt];
        % Soporte horizontal
        \foreach \xn in {1,2,...,\NUMTOTHOR}
        \draw[soporte] (\DISTSOPORTE * \xn, 0) -- +(\ANGSOPORTE:\LSOPORTE em);
        % Soporte vertical
        \foreach \yn in {1,2,...,\NUMTOTVER}
        \draw[soporte] (0,\DISTSOPORTE * \yn) -- +(\ANGSOPORTE:\LSOPORTE em);
        % Líneas auxiliares del peso
        \draw[linea auxiliar] (peso) -- (pesopar);
        \draw[linea auxiliar] (peso) -- (pesoperp);
        % Peso
        \draw[fuerza] (part) -- (peso);
        \node[below left=-5pt and 0pt, green!60!black] at (peso) {$\vvv{P}$};
        % Normal
        \draw[fuerza] (part) -- (normal);
        \node[above, green!60!black] at (normal) {$\vvv{N}$};
        % Componente paralela del peso
        \draw[componente fuerza] (part) -- (pesopar);
        \node[above right=0pt and -3pt,black!50] at (pesopar) {$\vvv{P}_{x'}$};
        % Componente perpendicular del peso
        % \draw[lightgray, ultra thin] (part) -- +(180+90-\ANGBARRA:\PESOPERPLINAUX);
        \draw[componente fuerza] (part) -- (pesoperp);
        \node[above left=0pt and -5pt,black!50] at (pesoperp) {$\vvv{P}_{y'}$};
        % Partícula
        \filldraw[particula] (part) circle;
        % YELLOW BACKGROUND
        \begin{scope}[on background layer]
          \node [background, fit= (bleft) (bright) (btop) (bbottom)] {};
        \end{scope}
      \end{tikzpicture}
      \caption{}
      \label{fig:el-barra2-fuerzas-par-perp}
    \end{subfigure}
    \caption{En la figura $(a)$ se representan los ejes cartesianos $x$ e $y$, con centro en $O$
      y los ejes paralelos y perpendiculares a la barra, con centro $O'$ en el extremo superior de
      la barra.
      En $(b)$ se representan las fuerzas que actúan sobre la partícula y las componentes
      del peso en los ejes fijos en la partícula que se desplaza. El ángulo $\theta$ se
      representa en la barra y en el peso con el fin de facilitar el cálculo de las componentes
      $P_{x'}$ y $P_{y'}$ de este último.}
\end{figure}
  
En el sistema de coordenadas que utilizamos, el peso tiene las componentes
\begin{align*}
  P_{x'} &= P\sin\theta = mg\sin\theta\\
  P_{y'} &= -P\cos\theta = -mg\cos\theta
\end{align*}

Para este ejemplo no nos sirven las leyes de equilibrio del ejemplo de la sección anterior, pues la
partícula va a caer. Ahora utilizamos la segunda ley de Newton, teniendo en cuenta que $a_{y'} = 0$
\[
  \sum_i \vvv{F}_i = m\vvv{a}
\]
\[
  \vvv{P} + \vvv{N} = m\vvv{a}
\]
\[
  (P_{x'}, -P_{y'}) + (0,N) = m\,(a_{x'}, 0)
\]
\[
  (mg\sin\theta, N-mg\cos\theta) = m(a_{x'}, 0)
\]

Igualando componente a componente obtenemos $a_{x'}$ y $N$.
\begin{align}
  &a_{x'} = g\sin\theta = 10\cdot\frac{2}{\sqrt{5}} = 4\sqrt{5}\,\unit{\metre\per\square\second}\\
  & N = mg\cos\theta = 0,1\cdot 10\cdot\frac{1}{\sqrt{5}} = \frac{\sqrt{5}}{5}\,\unit{\newton}
\end{align}
donde hemos utilizado las razones trigonométricas~\ref{eq:el-barra2-sin}
y~\ref{eq:el-barra2-cos}, además de aproximar la aceleración de la gravedad en la superficie
de la Tierra a $g\approx \SI{10}{\metre\per\square\second}$.

Como $a_{x'}$ es la única componente de la aceleración, en valor absoluto es su módulo
\begin{equation}\label{eq:el-barra2-newton-a}
  a = \sqrt{a_{x'}^2 + a_{y'}^2} = \sqrt{(4\sqrt{5})^2 + 0^2} = 4\sqrt{5}\,\si{\metre\per\square\second}
\end{equation}

%Las componentes del peso son
%\begin{align}
%  P_{x'} &= mg\sin\theta = 0,1\cdot 10\cdot\frac{2}{\sqrt{5}} = \frac{2\sqrt{5}}{5}\,\si{\newton}\\
%  P_{y'} &= -mg\cos\theta = -0,1\cdot 10\cdot\frac{1}{\sqrt{5}} = \frac{\sqrt{5}}{5}\,\si{\newton}
%\end{align}

Para poder comparar estos resultados con los que obtendremos después aplicando el principio de
D'Alembert dinámico, vamos a \emph{rotar} un ángulo $\theta=+\ang{60}$ el sistema de coordenadas
de ejes $x'$ e $y'$, para obtener la misma orientación que el sistema con ejes $x$ e $y$ horizontal
y vertical, respectivamente.

Rotaremos los ejes de coordenadas (no los vectores que se representan en este espacio).
Se trata pues, de una rotación pasiva. Ver figura~\ref{fig:el-barra2-rotacion-pasiva-60}.

\begin{figure}[ht]
  \def\scl{1.0}
  \pgfmathsetmacro{\ANGEJESPRIMA}{60}
  \pgfmathsetmacro{\LEJEXPRIMA}{2.5}
  \pgfmathsetmacro{\LEJEYPRIMA}{2.5}
  \pgfmathsetmacro{\LEJEX}{2.5}
  \pgfmathsetmacro{\LEJEY}{2.5}
  \pgfmathsetmacro{\RADIOANGULO}{30}
  % 
  \pgfmathsetmacro{\BGTOP}{3.2}
  \pgfmathsetmacro{\BGBOTTOM}{-2.7}
  \pgfmathsetmacro{\BGLEFT}{-.6}
  \pgfmathsetmacro{\BGRIGHT}{3.2}
  %
  \centering
  \begin{tikzpicture}[%
    scale=\scl,
    angulo/.style={%
      -{Latex[length=6pt, width=4pt, bend]}, shorten >=2pt, shorten <=2pt, draw=red
    },
    eje/.style={->},
    eje O/.style={eje, black, line width=1.2pt},
    eje O'/.style={eje, green!60!black, line width=1.2pt},
    background/.style={%
      line width=\bgborderwidth,
      draw=\bgbordercolor,
      fill=\bgcolor,
    },
    ]
    % COORDENADAS
    \coordinate (O) at (0,0);
    % Ejes
    \path (O) -- +(right:\LEJEX) coordinate (ejex);
    \path (O) -- +(up:\LEJEY) coordinate (ejey);
    \path (O) -- +(90-\ANGEJESPRIMA:\LEJEYPRIMA) coordinate (ejeyprima);
    \path (ejeyprima) -- +(90-\ANGEJESPRIMA:8pt) coordinate (etiqyprima);
    \path (O) -- +(360-\ANGEJESPRIMA:\LEJEXPRIMA) coordinate (ejexprima);
    \path (ejexprima) -- +(360-\ANGEJESPRIMA:8pt) coordinate (etiqxprima);
    \path (O) -- +(360-\ANGEJESPRIMA:\RADIOANGULO pt) coordinate (xprimaangulo);
    \path (O) -- +(90-\ANGEJESPRIMA:\RADIOANGULO pt) coordinate (yprimaangulo);
    % 
    \coordinate (btop) at (0,\BGTOP);
    \coordinate (bbottom) at (0,\BGBOTTOM);
    \coordinate (bleft) at (\BGLEFT,0);
    \coordinate (bright) at (\BGRIGHT,0);
    % DIBUJO
    % Ángulo x' -> x
    \draw[angulo] (xprimaangulo)
    arc [start angle=-\ANGEJESPRIMA, end angle=0, radius=\RADIOANGULO pt]
    node[right, pos=.3, red!80!black]  {$+\theta$};
    % Ángulo y' -> y
    \draw[angulo] (yprimaangulo)
    arc [start angle=90-\ANGEJESPRIMA, end angle=90, radius=\RADIOANGULO pt]
    node[above=3pt, pos=.3, red!80!black]  {$+\theta$};
    % Ejes x e y
    \draw[eje O] (O) -- (ejex);
    \node[right] at (ejex) {$x$};
    \draw[eje O] (O) -- (ejey);
    \node[above] at (ejey) {$y$};
    % Ejes x' e y'
    \draw[eje O'] (O) -- (ejexprima);
    \node[above=-5pt, green!60!black] at (etiqxprima) {$x'$};
    \draw[eje O'] (O) -- (ejeyprima);
    \node[right=-7pt, green!60!black] at (etiqyprima) {$y'$};
    %
    % YELLOW BACKGROUND
    \begin{scope}[on background layer]
      \node [background, fit= (bleft) (bright) (btop) (bbottom)] {};
    \end{scope}
  \end{tikzpicture}
  \caption{Rotación pasiva de los ejes paralelo y perpendicular a la barra $x'$ e $y'$, para
    transformarlos en los ejes de coordenadas horizontal y vertical $x$ e $y$.}
  \label{fig:el-barra2-rotacion-pasiva-60}
\end{figure}

Como consecuencia de esta rotación, las componentes cartesianas de los vectores $\vvv{N}$ y
$\vvv{a}$, medidos según los ejes $x'$ e $y'$, se convierten en sus componentes según los ejes
horizontal y vertical, $x$ e $y$.

La matriz de rotación pasiva en $\symbb{R}^2$ es
\[
  \mmm{M}
  =
  \begin{pNiceMatrix}
    \cos\theta & \sin\theta\\
    -\sin\theta & \cos\theta \\
  \end{pNiceMatrix}  
\]

Cuando se aplica esta transformación a la fuerza normal sobre la partícula expresado en el sistema
inclinado, obtenemos sus componentes en el sistema sin inclinar.
\[
  \begin{pNiceMatrix}
    N_x \\
    N_y
  \end{pNiceMatrix}
  =
  \begin{pNiceMatrix}
    \cos\theta & \sin\theta\\
    -\sin\theta & \cos\theta \\
  \end{pNiceMatrix}
  \begin{pNiceMatrix}
    N_{x'} \\
    N_{y'}
  \end{pNiceMatrix}   
   =
  \begin{pNiceMatrix}
    \cos\theta & \sin\theta\\
    -\sin\theta & \cos\theta \\
  \end{pNiceMatrix}
  \begin{pNiceMatrix}
    0 \\
    N
  \end{pNiceMatrix}
\]

Así, obtenemos
\begin{align}
  N_x &= N\sin\theta = \frac{\cancelout{\sqrt{5}}}{5}\cdot\frac{2}{\cancelout{\sqrt{5}}}
        = \SI{0,4}{\newton}\\
  N_y &= -N\cos\theta = -\frac{\cancelout{\sqrt{5}}}{5}\cdot\frac{1}{\cancelout{\sqrt{5}}}
        = -\SI{0,2}{\newton}
\end{align}

Hacemos lo mismo con el vector aceleración de la partícula
\[
  \begin{pNiceMatrix}
    a_x \\
    a_y
  \end{pNiceMatrix}
  =
  \begin{pNiceMatrix}
    \cos\theta & \sin\theta\\
    -\sin\theta & \cos\theta \\
  \end{pNiceMatrix}
  \begin{pNiceMatrix}
    a_{x'} \\
    a_{y'}
  \end{pNiceMatrix}    
  =
  \begin{pNiceMatrix}
    \cos\theta & \sin\theta\\
    -\sin\theta & \cos\theta \\
  \end{pNiceMatrix}
  \begin{pNiceMatrix}
    g\sin\theta \\
    0
  \end{pNiceMatrix} 
\]

Las componentes cartesianas de la aceleración en el sistema de referencia no inclinado son
%\begin{align}
%  \label{eq:el-barra2-newton-ax}
%  a_x &= g\sin\theta\cos\theta = 10\frac{2}{\sqrt{5}}\frac{1}{\sqrt{5}}
%        = \SI{4}{\metre\per\square\second}\\
%  \label{eq:el-barra2-newton-ay}
%  a_y &= -g\sin^2\theta = -10\,\left(\frac{2}{\sqrt{5}}\right)^2
%        = -10\cdot\frac{4}{5}
%        = -\SI{8}{\metre\per\square\second}
%\end{align}
\begin{equation}
  \vvv{a}\,
  \left\{
    \begin{array}{l}
      \label{eq:el-barra2-newton-a-dos}
      a_x = g\sin\theta\cos\theta = 10\frac{2}{\sqrt{5}}\frac{1}{\sqrt{5}}
      = \SI{4}{\metre\per\square\second}\\
      a_y = -g\sin^2\theta = -10\,\left(\frac{2}{\sqrt{5}}\right)^2
        = -10\cdot\frac{4}{5} = -\SI{8}{\metre\per\square\second}
    \end{array}
    \right.
\end{equation}

Antes de resolver el ejercicio mediante el principio de D'Alembert, calculemos el tiempo
que tardará en llegar al suelo, si la partícula se suelta desde lo alto de la barra, a
\SI{2}{\metre} de altura.

La partícula cae con una aceleración constante $a = g\sin\theta$ en el sentido de descenso
de la barra
\[
  x' = x'_0 + v_0 t + \frac{1}{2} a t^2
\]

Cuando recorre toda la longitud de la barra, tenemos
\[
  L = x'_0 + v_0 t + \frac{1}{2} a t^2
\]

Recopilamos todos los datos que necesitamos:
\begin{align*}
  x'_0 &= \SI{0}{\metre}\\
  v_0 &= \SI{0}{\metre\per\second}\\
  g &\approx \SI{10}{\metre\per\square\second}\\
  L &= \sqrt{2^2 + 1^2} = \sqrt{5}\,\si{\metre}
\end{align*}

Aplicando estos datos
\[
  \sqrt{5} = \frac{1}{2} \left(10\cdot\frac{2}{\sqrt{5}}\right)\,t^2
\]
obtenemos
\begin{equation}\label{eq:el-barra2-newton-tiempo}
  t = \frac{1}{\sqrt{2}}\,\si{\second}
\end{equation}

\par
\item Aplicando el \emph{principio de D'Alembert dinámico}:

  Tenemos una fuerza de ligadura, $\vvv{N}$, que es la fuerza normal que ejerce la barra
  sobre la partícula. Nos queda una única fuerza aplicada, $\vvv{P}$, que es el peso de la
  partícula, $\vvv{P} = (0,-mg)$.

  El principio de D'Alembert dinámico es
  \[
    \sum_i \left(\vvv{\dot{p}}_i - \vvv{F}_i^{(a)}\right) \cdot \delta\vvv{r}_i = 0
  \]
  Lo aplicamos a la partícula
  \[
    (m\vvv{a} - \vvv{P})\cdot \delta\vvv{r} = 0
  \]  
  \[
    [m\,(a_x,a_y) - (0,-mg)]\cdot \delta\vvv{r} = 0
  \]
  \[
    m\,(a_x,a_y+g)\cdot \delta\vvv{r} = 0
  \]
  \begin{equation}\label{eq:el-barra2-PD-temp}
    (a_x,a_y+g)\cdot \delta\vvv{r} = 0
  \end{equation}
  
  Para estudiar la variación $\delta\vvv{r}$ necesitamos representar paramétricamente la barra
  por donde desliza la partícula, ver figura~\ref{fig:el-barra2-r}.
  \begin{figure}[ht]
    % Escala
    \def\scl{.75}
    % 
    %\centering
    %\begin{subfigure}[b]{.42\textwidth}
    % Barra
    \pgfmathsetmacro{\LBARRA}{5.5}
    \pgfmathsetmacro{\ANGBARRA}{60}
    \pgfmathsetmacro{\RADIOANGBARRA}{1.05}
    \pgfmathsetmacro{\LHORBARRA}{\LBARRA * cos(\ANGBARRA)}
    \pgfmathsetmacro{\LVERBARRA}{\LBARRA * sin(\ANGBARRA)}
    % Posición de la partícula en la barra contanto desde el suelo
    \pgfmathsetmacro{\LPART}{.5*\LBARRA}
    % Suelo
    \pgfmathsetmacro{\LSUELO}{\LHORBARRA + .8}
    \pgfmathsetmacro{\LPARED}{\LVERBARRA + .7}
    % Material de soporte suelo y pared
    \pgfmathsetmacro{\DISTSOPORTE}{.3}
    \pgfmathsetmacro{\LSOPORTE}{1.1}
    \pgfmathsetmacro{\ANGSOPORTE}{230}
    \pgfmathsetmacro{\NUMTOTHOR}{int(\LSUELO / \DISTSOPORTE)}
    \pgfmathsetmacro{\NUMTOTVER}{int(\LPARED / \DISTSOPORTE)}
    % Peso
    % Módulo del peso
    \pgfmathsetmacro{\LPESO}{2.0}
    % Componente del peso paralela a la barra
    \pgfmathsetmacro{\PESOPAR}{\LPESO * sin(\ANGBARRA)}
    % Componente del peso perpendicular a la barra
    \pgfmathsetmacro{\PESOPERP}{\LPESO * cos(\ANGBARRA)}
    \pgfmathsetmacro{\PESOPERPLINAUX}{1.35*\PESOPERP}
    % Líneas auxiliares
    % Normal
    \pgfmathsetmacro{\LNORMAL}{\PESOPERP}
    % Ejes x' e y'
    \pgfmathsetmacro{\LXPRIMA}{1.2*\PESOPAR}
    \pgfmathsetmacro{\LYPRIMA}{2.3*\LNORMAL}
    % Ejes x e y
    \pgfmathsetmacro{\LX}{.9}
    \pgfmathsetmacro{\LY}{.9}
    %    
    \pgfmathsetmacro{\BGTOP}{5.9}
    \pgfmathsetmacro{\BGBOTTOM}{-1.5}
    \pgfmathsetmacro{\BGLEFT}{-1.9}
    \pgfmathsetmacro{\BGRIGHT}{5.0}
    %
    \centering
    \begin{tikzpicture}[%
        scale=\scl,
        baseline,
        punto fuerza/.style={fill=green!60!black, draw=green!60!black, radius=1.1pt},
        punto barra/.style={fill=black, draw=black, radius=.5pt},
        angulo/.style={draw=green!50!black, fill=green!20},
        every node/.style={black,font=\small},
        barra/.style={black!50},
        vector posicion/.style={-{Latex[length=8pt,width=4.5pt]}, line width=1.4pt, red!80!black},
        fuerza/.style={-{Latex[length=6pt,width=4pt]}, shorten >=-2pt, line width=1.0pt, green!60!black},
        fuerza ligadura/.style={-{Latex[length=6pt,width=4pt]}, shorten >=-2pt, line width=1.0pt, black!25},
        componente fuerza/.style={-{Latex[length=8pt,width=4.5pt]}, shorten >=-2pt, line width=1.4pt, black!60},
        particula/.style={fill=red, draw=black, radius=2.5pt},
        eje/.style={-{>}, black, line width=.8pt},
        limite/.style={black!50},
        soporte/.style={black!25, ultra thin},
        linea auxiliar/.style={lightgray, ultra thin},
        background/.style={%
          line width=\bgborderwidth,
          draw=\bgbordercolor,
          fill=\bgcolor,
        },
        ]
        % COORDENADAS
        \coordinate (O) at (0,0);
        \coordinate (xsuelo) at ($(O) + (\LSUELO, 0)$);
        \coordinate (ypared) at ($(O) + (0,\LPARED)$);
        \coordinate (barra1) at ($(O) + (\LHORBARRA, 0)$);
        \coordinate (barra2) at ($(O) + (0,\LVERBARRA)$);
        \coordinate (peso) at (part);
        \path (barra1) -- +(180-\ANGBARRA:\LPART) coordinate (part);
        \path (part) -- +(down:\LPESO) coordinate (peso);
        \path (part) -- +(360-\ANGBARRA:\PESOPAR) coordinate (pesopar);
        \path (part) -- +(270-\ANGBARRA:\PESOPERP) coordinate (pesoperp);
        \path (part) -- +(90-\ANGBARRA:\LNORMAL) coordinate (normal);
        \path (part) -- +(90-\ANGBARRA:\LYPRIMA) coordinate (yprima);
        \path (part) -- +(360-\ANGBARRA:\LXPRIMA) coordinate (xprima);
        % 
        \coordinate (btop) at (0,\BGTOP);
        \coordinate (bbottom) at (0,\BGBOTTOM);
        \coordinate (bleft) at (\BGLEFT,0);
        \coordinate (bright) at (\BGRIGHT,0);
        % DIBUJO
        % Medidas
        % pared
        \draw[linea auxiliar] (O) -- +(left:1.3) coordinate (LO);
        \draw[linea auxiliar] (barra2) -- +(left:1.3) coordinate (Lbarra2);
        \path (O) -- node[left=.25] {$\SI{2}{\metre}$} (barra2);
        \draw[{Straight Barb}-, lightgray] ($(LO)!.32!(O)$) -- +(up:2.1);
        \draw[{Straight Barb}-, lightgray] ($(Lbarra2)!.32!(barra2)$) -- +(down:2.1);
        % suelo
        \draw[lightgray, ultra thin] (O) -- +(down:1.1) coordinate (BO);
        \draw[lightgray, ultra thin] (barra1) -- +(down:1.1) coordinate (Bbarra1);
        \path (O) -- node[below=.3] {$\SI{1}{\metre}$} (barra1);
        \draw[{Straight Barb}-, lightgray] ($(BO)!.31!(O)$) -- +(right:1.0);
        \draw[{Straight Barb}-, lightgray] ($(Bbarra1)!.31!(barra1)$) -- +(left:1.0);
        % Suelo
        \draw[soporte] (O) -- (xsuelo);
        % Pared
        \draw[soporte] (O) -- (ypared);
        % Barra
        \draw[barra] (barra1) -- (barra2);
        % Extremos de la barra
        \filldraw[punto barra] (barra1) circle;
        \filldraw[punto barra] (barra2) circle;        
        \node[above right] at (barra1) {$B(1,0)\,\si{\metre}$};
        \node[above right] at (barra2) {$A(0,2)\,\si{\metre}$};        
        % Partícula
        \filldraw[fill=red,draw=black] (part) circle[radius=2pt];
        % Soporte horizontal
        \foreach \xn in {1,2,...,\NUMTOTHOR}
        \draw[soporte] (\DISTSOPORTE * \xn, 0) -- +(\ANGSOPORTE:\LSOPORTE em);
        % Soporte vertical
        \foreach \yn in {1,2,...,\NUMTOTVER}
        \draw[soporte] (0,\DISTSOPORTE * \yn) -- +(\ANGSOPORTE:\LSOPORTE em);
        % Vector r de la partícula
        \draw[vector posicion] (O) -- (part);
        % Ejes x e y
        \draw[eje] (O) -- +(right:\LX) node[below=2pt] {$x$};
        \draw[eje] (O) -- +(up:\LY) node[left=2pt] {$y$};
        \filldraw[fill=black, draw=black] (O) circle[radius=1pt];
        \node[below left] at (O) {$O$};
        % Peso
        \draw[fuerza] (part) -- (peso);
        \node[right, green!60!black] at (peso) {$\vvv{P}$};
        % Normal
        \draw[fuerza ligadura] (part) -- (normal);
        \node[above, black!25] at (normal) {$\vvv{N}$};
        % Partícula
        \filldraw[particula] (part) circle;
        % YELLOW BACKGROUND
        \begin{scope}[on background layer]
          \node [background, fit= (bleft) (bright) (btop) (bbottom)] {};
        \end{scope}
      \end{tikzpicture}
      \caption{En la figura se representan las coordenadas de los extremos de la barra,
        respecto del sistema de coordenadas con centro en $O$. Además se representa en
        color rojo la posición $\vvv{r}$ de la partícula respecto a este mismo sistema de
        referencia.}
      \label{fig:el-barra2-r}
  \end{figure}

  La ecuación de la recta que representa a la barra debe tener la forma $y=mx+n$.
  El punto $B(1,0)$ debe pertenecer a la recta
  \[
    2 = m\cdot 0 + n
    ;\hspace{1em}
    n = 2
  \]
  La recta se podrá escribir temporalmente como $y=mx+2$ y el punto $A(0,2)$ debe pertenecer
  también a la recta
  \[
    0 = m\cdot 1 + 2
    ;\hspace{1em}
    m = -2
  \]
  La recta que pasa por los dos puntos $A$ y $B$ es $y=-2x+2$.

  La partícula se mueve por una dimensión. Así, necesitaremos un solo parámetro $q$.
  Podemos especificar las coordenadas $x$ e $y$ de la posición de la partícula a través del
  parámetro, de muchas formas diferentes. Una de ellas puede ser
  \begin{equation}\label{eq:el-barra2-parametrizacion-r}
    \vvv{r}\,
    \left\{
      \begin{array}{l}
        x = q\\
        y = -2q+2
      \end{array}
      \right.
  \end{equation}
  de manera que si variamos $q$, estamos variando $\vvv{r}$.
  \[
    \delta\vvv{r}\,
    \left\{
      \begin{array}{l}
        \delta x = \dfrac{\partial x}{\partial q}\,\delta q
        = \dfrac{\partial q}{\partial q}\,\delta q = \delta q\\[2ex]
        \delta y = \dfrac{\partial y}{\partial q}\,\delta q
        = \dfrac{\partial (-2q+2)}{\partial q}\,\delta q = -2\delta q
      \end{array}
      \right.
  \]

  La variación virtual de la posición de la partícula es
  \begin{equation}\label{eq:el-barra2-dr}
    \delta\vvv{r} = (\delta x, \delta y) = (1, -2)\,\delta q
  \end{equation}

  Sustituyendo el resultado~(\ref{eq:el-barra2-dr}) en la ecuación~(\ref{eq:el-barra2-PD-temp})
  \[
    (a_x,a_y+g)\cdot (1,-2)\,\delta q = 0
  \]
  \[
    [a_x-2(a_y+g)]\,\delta q = 0
  \]
  Como $\delta q$ no puede anularse, el otro factor debe ser cero
  \begin{equation}\label{eq:el-barra2-resultado-temp}
    a_x -2 (a_y + g) = 0
  \end{equation}

  Este resultado nos da la relación entre las componentes de la aceleración $a_x$ y $a_y$, pero no
  parece coincidir con los que se obtuvo aplicando las leyes de Newton
  (\ref{eq:el-barra2-newton-a}) y (\ref{eq:el-barra2-newton-tiempo}), pero sigamos desarrollando
  las expresiones que tenemos.
  Utilizaremos la parametrización~(\ref{eq:el-barra2-parametrizacion-r}) para obtener las
  componentes de la aceleración en función del parámetro, derivando dos veces con respecto del
  tiempo
  \begin{equation}\label{eq:el-barra2-r-dot}
    \vvv{v} = \vvv{\dot{r}}\,
    \left\{
      \begin{array}{l}
        v_x = \dot{x} = \dot{q}\\
        v_y = \dot{y} = -2\dot{q}
      \end{array}
    \right.
  \end{equation}
  \begin{equation}\label{eq:el-barra2-r-dotdos}
    \vvv{a} = \vvv{\ddot{r}}\,
    \left\{
      \begin{array}{l}
        a_x = \ddot{x} = \ddot{q}\\
        a_y = \ddot{y} = -2\ddot{q}
      \end{array}
    \right.
  \end{equation}

  Reescribimos~(\ref{eq:el-barra2-resultado-temp})
  \[
    \ddot{q} - 2(-2\ddot{q} + g) = 0
  \]
  de donde obtenemos
  \begin{equation}\label{eq:el-barra2-q-dotdot}
    \ddot{q} = \dfrac{2}{5}\,g = \dfrac{2}{5}\,10 = \SI{4}{\metre\per\square\second}
  \end{equation}
  
  Las coordenadas cartesianas de la aceleración quedan
  \begin{equation}
    \vvv{a}\,
    \left\{
      \begin{array}{l}
        \label{eq:el-barra2-PD-a}
        a_x = \SI{4}{\metre\per\square\second}\\
        a_y = -\SI{8}{\metre\per\square\second}
      \end{array}
    \right.
  \end{equation}
  que coinciden con las que se obtuvo en~(\ref{eq:el-barra2-newton-a}), aplicando las leyes de
  Newton.

  Ahora nos podría interesar cómo varía el parámetro $q$ con el tiempo. Para averiguarlo,
  integramos~(\ref{eq:el-barra2-q-dotdot})
  \[
    \int \ddot{q}(t)\,dt = \int 4\,dt
  \]
  lo que nos da
  \[
    \dot{q}(t) = 4t + C_1
  \]
  Volviendo a integrar
  \[
    \int\dot{q}(t)\,dt = \int (4t + C_1)\,dt
  \]
  \[
    q(t) = 4\dfrac{t^2}{2} + C_1 t + C_2
  \]
  \begin{equation}\label{eq:el-barra2-q-c1-c2}
    q(t) = 2t^2 + C_1 t + C_2
  \end{equation}

  Necesitamos calcular las dos constantes de integración, $C_1$ y $C_2$.
  Sabemos de~(\ref{eq:el-barra2-parametrizacion-r}) que $x=q$, y como $x_0 = x(0) = 0$,
  por tanto, $q(0)$ también debe anularse
  \[
    q(0) = C_2
    ;\hspace{1em}
    C_2 = 0
  \]

  Así, la expresión~(\ref{eq:el-barra2-q-c1-c2}) se convierte en
  \[
    q(t) = 2t^2 + C_1 t
  \]

  Derivamos esta expresión y obtenemos
  \[
    \dot{q}(t) = 4t + C_1
  \]
  
  Sabemos que la partícula estaba inicialmente en reposo, $v_x=v_y=0$. Ahora nos fijamos
  en~(\ref{eq:el-barra2-r-dot}) y deducimos que $\dot{q}(0) = 0$, y por tanto
  \[
    \dot{q}(0) = C_1 = 0
  \]

  Concluimos que
  \begin{equation}
    q(t) = 2t^2
  \end{equation}
  y de acuerdo con~(\ref{eq:el-barra2-parametrizacion-r})
  \begin{equation}\label{eq:el-barra2-r-t}
    \vvv{r}\,
    \left\{
      \begin{array}{l}
        x = q = 2t^2\\
        y = -2q+2 = -2\cdot 2t^2 + 2 = -4t^2 + 2
      \end{array}
    \right.
  \end{equation}
  
  Cuando la partícula llegue al final de la barra, sus coordenadas serán $B(1,0)$,
  por tanto $y = 0$, de donde se deduce de~(\ref{eq:el-barra2-r-t}) que
  $0 = -4t^2 + 2$ y $t = 1/\sqrt{2}\,\si{\second}$, lo que coincide con el tiempo de
  caída~(\ref{eq:el-barra2-newton-tiempo}) de la partícula a través de las leyes de Newton.
\end{itemize}

Conforme los sistemas se vuelven más complicados (con más ligaduras), la aplicación de las leyes de
Newton se convierte en un calvario, mientras que parametrizando el vector de posición, utilizando
unas variables que llamaremos \emph{coordenadas generalizadas} $q_i$ es un más sencillo.
Pero, no con el principio de D'Alembert, sino por medio de las ecuaciones de Euler-Lagrange,
que no dan tanto rodeo.

\section{Ecuaciones de Euler-Lagrange}

\subsection{Deducción sencilla}
Para que la explicación sea lo más sencilla posible, vamos a deducirlas para una sola partícula en
el caso de que se necesite únicamente una coordenada generalizada $q$. En cuanto se entienda este
desarrollo, sería casi inmediato generalizarlo para más partículas y más coordenadas generalizadas.
\subsubsection{Consideraciones generales a tener en cuenta}
Para deducir las ecuaciones de Euler-Lagrange es muy importante tener en cuenta, entre otras,
las siguientes consideraciones:
\begin{itemize}
\item La posición de una partícula sólo puede depender de $q$ y de $t$, pero no de la velocidad
  generalizada $\dot{q}$
  \begin{equation}
    \vvv{r} = \vvv{r}(q,t)
  \end{equation}
  
  El que la posición no dependa de $\dot{q}$ implicará que la derivada parcial de $\vvv{r}$ o, de
  cualquier derivada de esta, con respecto de $\dot{q}$ vale cero. Así podríamos escribir
  \begin{equation}
    \frac{\partial\vvv{r}}{\partial\dot{q}} = 0
    ;\hspace{1em}
    \frac{\partial}{\partial\dot{q}}\left(\frac{\partial\vvv{r}}{\partial q}\right) = 0
    ;\hspace{1em}
    \frac{\partial}{\partial\dot{q}}\left(\frac{\partial\vvv{r}}{\partial t}\right) = 0
  \end{equation}
  En cambio, la velocidad $\vvv{v}$ sí puede depender de la velocidad generalizada $\dot{q}$, así que
  la siguiente derivada tiene porqué anularse
  \[
    \frac{\partial\vvv{v}}{\partial\dot{q}} \neq 0
    \hspace{2em}
    \text{(en general)}
  \]

\item En algunos momentos puede ser útil aplicar el \emph{teorema de Clairaut o de Schwartz},
  que nos permite intercambiar el orden de las derivadas parciales, a condición de que la función
  que se deriva \emph{se comporte bien}, es decir, que sea continua y \emph{no presente picos}
  (que su derivadas parciales existan y sean continuas).
  Por ejemplo, puesto que el vector de posición de una partícula \emph{se comporta bien}, podríamos
  escribir
  \begin{equation}
    \frac{\partial}{\partial q}\left(\frac{\partial\vvv{r}}{\partial t}\right)
    = \frac{\partial}{\partial t}\left(\frac{\partial\vvv{r}}{\partial q}\right)
  \end{equation}

\item El desplazamiento virtual $\delta\vvv{r}$ es cualquier variación infinitesimal de la posición
  de una partícula, consistente con las ligaduras impuestas al sistema, pero sin que varíe el
  tiempo.

  Recordemos que para obtener el desplazamiento virtual de una partícula, debíamos parametrizar
  sus coordenadas cartesianas. Pero esta parametrización no podía ser cualquiera, pues debía
  respetar las ligaduras impuestas al sistema.
  
  Por ejemplo, en el caso de la partícula que se deslizaba sobre la barra en la sección
  anterior, el desplazamiento virtual estaba restringido al movimiento sobre la barra, por eso
  tuvimos en cuenta la ecuación de la recta que definía la barra
  \[
    \delta\vvv{r} = (q, -2q+2)
  \]
  En este ejemplo, la coordenada generalizada era el parámetro $q$.

  Si, por ejemplo, una partícula se pudiera mover únicamente a lo largo de una circunferencia de
  radio $R$, el desplazamiento virtual sería
  \[
    \delta\vvv{r} = (R\cos\theta, R\sin\theta)
  \]
  En este caso, la coordenada generalizada sería el ángulo girado $\theta$.

  Si en la parametrización entrara el tiempo, por ejemplo
  \[
    x = 2q + t
    ;\hspace{1em}
    y = q
  \]
  se trataría de un sistema no conservativo.
  
  Es muy importante recalcar que un desplazamiento virtual ($\delta\vvv{r}$) se distingue de otro
  real ($d\vvv{r}$) en que en el primero no varía el tiempo y en el último sí. El desplazamiento
  virtual se podría entender como la capacidad que tendrían las partículas de un sistema para
  moverse.
  \end{itemize}

\subsubsection{Cuatro identidades preliminares}
Para no romper el hilo de la deducción de las ecuaciones de Euler-Lagrange, empezaremos obteniendo
cuatro identidades que necesitaremos en algún momento del desarrollo:
\begin{enumerate}
\item Primera identidad
  \begin{equation}\label{eq:el-primera-identidad}
    \vvv{v}
    = \frac{\partial\vvv{r}}{\partial q}\,\dot{q}
    + \frac{\partial\vvv{r}}{\partial t}    
  \end{equation}

  Demostración:
  Podemos calcular la velocidad de la partícula como derivada de la posición con respecto del
  tiempo y aplicar la regla de la cadena porque, a su vez, la posición puede depender de $q$
  y de $t$.
  \[
    \vvv{v}
    = \frac{d\vvv{r}}{dt}
    = \frac{\partial\vvv{r}}{\partial q}\,\frac{dq}{dt}
    + \frac{\partial\vvv{r}}{\partial t}\,\frac{dt}{dt}
    = \frac{\partial\vvv{r}}{\partial q}\,\dot{q}
    + \frac{\partial\vvv{r}}{\partial t}
  \]

\item Segunda identidad
  \begin{equation}\label{eq:el-segunda-identidad}
    \frac{\partial\vvv{r}}{\partial q} = \frac{\partial\vvv{v}}{\partial\dot{q}}
  \end{equation}

  Demostración:
  Empezaremos por el segundo miembro de la igualdad, sustituyendo $\vvv{v}$ por el resultado de la
  identidad anterior. Después aplicaremos la regla de la cadena, considerando que $\vvv{r}$ depende
  como mucho de $q$ y de $t$, pero no de $\dot{q}$
  \begin{align*}
    \frac{\partial\vvv{v}}{\partial\dot{q}}
    &= \frac{\partial}{\partial\dot{q}}\left(\frac{\partial\vvv{r}}{\partial q}\,\dot{q}
      + \frac{\partial\vvv{r}}{\partial t}\right)
    = \frac{\partial}{\partial\dot{q}}\left(\frac{\partial\vvv{r}}{\partial q}\right)\dot{q}
    +\frac{\partial\vvv{r}}{\partial q}\cdot\frac{\partial\dot{q}}{\partial\dot{q}}
    + \frac{\partial}{\partial\dot{q}}\left(\frac{\partial\vvv{r}}{\partial t}\right)\\
    &= 0\cdot\dot{q}
    +\frac{\partial\vvv{r}}{\partial q}\cdot 1
      + \frac{\partial}{\partial t}\left(\frac{\partial\vvv{r}}{\partial\dot{q}}\right)
      = \frac{\partial\vvv{r}}{\partial q}
      + \frac{\partial}{\partial t}\left(0\right)
      = \frac{\partial\vvv{r}}{\partial q}
  \end{align*}
  donde se ha aplicado el \emph{teorema de Schwartz}, al intercambiar el orden de derivadas
  parciales.
  Además, la derivada parcial de $\vvv{r}$ con respecto de $\dot{q}$ se anula porque la posición
  no depende de $\dot{q}$, como ya sabíamos.
  
\item Tercera identidad
  \begin{equation}\label{eq:el-tercera-identidad}
    \frac{d}{dt}\left(\frac{\partial\vvv{r}}{\partial q}\right)
    = \frac{\partial\vvv{v}}{\partial q}
  \end{equation}

  Esta identidad nos demuestra que podemos intercambiar la derivada parcial y la total
  \[
    \frac{d}{dt}\left(\frac{\partial\vvv{r}}{\partial q}\right)
    = \frac{\partial}{\partial q}\left(\frac{d\vvv{r}}{d t}\right)
    = \frac{\partial\vvv{v}}{\partial q}
  \]
  
  Demostración:
  Aplicamos la regla de la cadena a $\partial\vvv{r}/\partial q$, teniendo en cuenta que
  puede depender de $q$ y de $t$
  \begin{align*}
  \frac{d}{dt}\left(\frac{\partial\vvv{r}}{\partial q}\right)
  &= \frac{\partial}{\partial q}\left(\frac{\partial\vvv{r}}{\partial q}\right)\cdot\frac{dq}{dt}
    + \frac{\partial}{\partial t}\left(\frac{\partial\vvv{r}}{\partial q}\right)\cdot\frac{dt}{dt}
    = \frac{\partial}{\partial q}\left(\frac{\partial\vvv{r}}{\partial q}\right)\cdot\dot{q}
    + \frac{\partial}{\partial q}\left(\frac{\partial\vvv{r}}{\partial t}\right)\\
    &= \frac{\partial}{\partial q}\left(\frac{\partial\vvv{r}}{\partial q}\,\dot{q}\right)
      + \frac{\partial}{\partial q}\left(\frac{\partial\vvv{r}}{\partial t}\right)
      = \frac{\partial}{\partial q}
      \left(\frac{\partial\vvv{r}}{\partial q}\,\dot{q} + \frac{\partial\vvv{r}}{\partial t}\right)\\
      &= \frac{\partial\vvv{v}}{\partial q}
\end{align*}
donde hemos tenido en cuenta que $\dot{q}$ es independiente de $q$, por lo que puede entrar en la
derivada parcial de $q$ y, en el último paso, hemos aplicado la primera
identidad~(\ref{eq:el-primera-identidad}).

\item Cuarta identidad
  \begin{equation}\label{eq:el-cuarta-identidad}
    m\vvv{v}\cdot\frac{\partial\vvv{v}}{\partial a}
    = \frac{\partial}{\partial a}\left(\frac{1}{2}mv^2\right)
  \end{equation}
  donde $a$ es una variable arbitraria.

  Demostración:
  Partimos de la derivada parcial del producto escalar de un vector por sí mismo, con respecto de
  una variable arbitraria $a$ y tenemos en cuenta que el producto escalar es conmutativo
  \[
    \frac{\partial }{\partial a}\left(\vvv{v}\cdot\vvv{v}\right)
    = \frac{\partial\vvv{v}}{\partial a}\cdot\vvv{v}
    + \vvv{v}\cdot\frac{\partial\vvv{v}}{\partial a}
    = 2\vvv{v}\cdot\frac{\partial\vvv{v}}{\partial a}
  \]
  
  Despejamos el producto escalar del segundo miembro
  \[
    \vvv{v}\cdot\frac{\partial\vvv{v}}{\partial a}
    = \dfrac{1}{2}\,\frac{\partial}{\partial a}
    \left(\vvv{v}\cdot\vvv{v}\right)
    = \dfrac{1}{2}\,\frac{\partial}{\partial a}\left(v^2\right)
  \]
  
  Finalmente, multiplicamos por la masa $m$ ambos miembros e introducimos las constantes
  $1/2$ y $m$ dentro de la derivada parcial
  \[
    m\vvv{v}\cdot\frac{\partial\vvv{v}}{\partial a}
    = \frac{\partial}{\partial a}\left(\frac{1}{2}mv^2\right)
  \]
\end{enumerate}

\subsubsection{Desarrollo}
Partimos del principio de D'Alembert dinámico
\begin{equation}\label{eq:el-PD-dinamico-dos}
  \sum_{i=1}^N \left(\dot{\vvv{p}}_i - \vvv{F}_i^{(a)}\right)\cdot \delta\vvv{r}_i
\end{equation}

Como sólo tenemos una partícula, el sumatorio se reduce a un único término, y para simplificar
la notación, consideremos que $\vvv{F}$ es la fuerza aplicada a la partícula (recordemos que
no se tienen en cuenta las fuerzas de ligadura)
\[
  \left(\dot{\vvv{p}} - \vvv{F}\right)\cdot \delta\vvv{r} = 0
\]
\[
  \dot{\vvv{p}}\cdot\delta\vvv{r} = \vvv{F}\cdot\delta\vvv{r}
\]

Parametrizaremos el vector posición de la partícula, de manera que dependa de la coordenada
generalizada y quizás del tiempo\footnotemark{}.
\footnotetext{Recordemos que si entraba el tiempo en la parametrización, se trataría de un sistema
  no conservativo.}
Esta parametrización no puede ser cualquiera, pues debe respetar las ligaduras impuestas al
sistema (en este caso, a la partícula). Llamaremos $q$ a este parámetro, que representa el único
grado de libertad que tiene la partícula.

Para obtener el desplazamiento virtual, aplicamos la regla de la cadena
\[
  \delta\vvv{r} = \frac{\partial\vvv{r}}{\partial q}\,\partial q
\]

Además, recordando que $\vvv{p} = m\vvv{v}$ y $\dot{\vvv{p}} = m\dot{\vvv{v}}$
\[
  m\vvv{\dot{v}} \cdot \frac{\partial\vvv{r}}{\partial q}\,\delta q
  = \vvv{F}\cdot\frac{\partial\vvv{r}}{\partial q}\,\delta q
\]

Definimos la \emph{fuerza generalizada} $Q$ como
\[
  Q \equiv \vvv{F}\cdot\frac{\partial\vvv{r}}{\partial q}
\]
De momento, consideraremos que es un nombre sin más; cuando se pongan ejemplos, veremos
porqué la llamamos así.

\begin{equation}\label{eq:el-EL-temp01}
  m\dot{\vvv{v}}\cdot\frac{\partial\vvv{r}}{\partial q}\,\delta q = Q\,\delta q
\end{equation}

Nos fijamos en el producto escalar del primer miembro de la igualdad anterior
(sin tener en cuenta el desplazamiento virtual).
Queremos conseguir que la derivada con respecto al tiempo ocupe toda esta expresión.
Empezamos introduciendo la constante $m$ en la derivada
\[
  m\dot{\vvv{v}}\cdot\frac{\partial\vvv{r}}{\partial q}
  = m\frac{dv}{dt}\cdot\frac{\partial\vvv{r}}{\partial q}
  = \frac{d}{dt}\left(m\vvv{v}\right)\cdot\frac{\partial\vvv{r}}{\partial q}
\]

Escribimos la expresión tal y como la queremos obtener y la desarrollamos
\begin{align*}
  \frac{d}{dt}\left[m\vvv{v}\cdot\frac{\partial\vvv{r}}{\partial q}\right]
  &= \frac{d}{dt}\left(m\vvv{v}\right)\cdot\frac{\partial\vvv{r}}{\partial q}
    + m\vvv{v}\cdot\frac{d}{dt}\left(\frac{\partial\vvv{r}}{\partial q}\right)
\end{align*}

Ahora despejamos y tenemos en cuenta~(\ref{eq:el-segunda-identidad}),
(\ref{eq:el-tercera-identidad}) y~(\ref{eq:el-cuarta-identidad})
\begin{align*}
  \frac{d}{dt}\left(m\vvv{v}\right)\cdot\frac{\partial\vvv{r}}{\partial q}
  &= \frac{d}{dt}\left[m\vvv{v}\cdot\frac{\partial\vvv{r}}{\partial q}\right]
    - m\vvv{v}\cdot\frac{d}{dt}\left(\frac{\partial\vvv{r}}{\partial q}\right)
    = \frac{d}{dt}\left[m\vvv{v}\cdot\frac{\partial\vvv{v}}{\partial\dot{q}}\right]
    - m\vvv{v}\cdot\frac{\partial\vvv{v}}{\partial q}\\
    &= \frac{d}{dt}\left[\frac{\partial}{\partial\dot{q}}\left(\frac{1}{2}mv^2\right)\right]
    - \frac{\partial}{\partial q}\left(\frac{1}{2}mv^2\right)
\end{align*}

Llamando $T$ a la energía cinética
\[
  \frac{d}{dt}\left(m\vvv{v}\right)\cdot\frac{\partial\vvv{r}}{\partial q}
  = \frac{d}{dt}\left(\frac{\partial T}{\partial\dot{q}}\right)
  - \frac{\partial T}{\partial q}
\]

Recuperamos la fórmula~\ref{eq:el-EL-temp01} sustituyendo el resultado anterior
\[
  \left[\frac{d}{dt}\left(\frac{\partial T}{\partial\dot{q}}\right)
    - \frac{\partial T}{\partial q}\right]\,\delta q
  = Q\delta q
\]

Reorganizando y sacando factor común $\delta q$
\[
  \left\{
    \left[\frac{d}{dt}\left(\frac{\partial T}{\partial\dot{q}}\right)\right] - Q
  \right\}\delta q = 0
\]

Como $\delta q$ no puede ser cero, obtenemos una primera versión de las ecuaciones de
Euler-Lagrange
\begin{equation}\label{eq:el-EL-primera-version}
  \frac{d}{dt}\left(\frac{\partial T}{\partial\dot{q}}\right) - \frac{\partial T}{\partial q}
  = \vvv{F}\cdot\frac{\partial\vvv{r}}{\partial q}
\end{equation}
Esta ecuación sólo es válida para una partícula con un grado de libertad.
La fuerza $\vvv{F}$ no tiene porqué ser conservativa, podría ser también una fuerza de fricción.

\subsubsection{Caso conservativo}
Pero en Física Teórica la fuerza $\vvv{F}$ suele ser conservativa, esto es, la fuerza proviene
de un potencial
\[
  \vvv{F} = -\vvv{\nabla} V
  = -\,\left(
    \frac{\partial V}{\partial x}, \frac{\partial V}{\partial y}, \frac{\partial V}{\partial z}
  \right)
\]

Calculamos el segundo miembro de la primera versión~\ref{eq:el-EL-primera-version}
de las ecuaciones de Euler-Lagrange
\begin{align*}
  \vvv{F}\cdot\frac{\partial\vvv{r}}{\partial q}
  &= -\left(
    \frac{\partial V}{\partial x}, \frac{\partial V}{\partial y}, \frac{\partial V}{\partial z}
  \right)
  \cdot
  \left(
    \frac{\partial x}{\partial q}, \frac{\partial y}{\partial q}, \frac{\partial z}{\partial q}
  \right)
  = -\left(
    \frac{\partial V}{\partial x}\,\frac{\partial x}{\partial q}
    + \frac{\partial V}{\partial y}\,\frac{\partial y}{\partial q}
    + \frac{\partial V}{\partial z}\,\frac{\partial z}{\partial q}
    \right)\\
  &= -\frac{\partial V}{\partial q}
\end{align*}

Sustituimos en~(\ref{eq:el-EL-primera-version})
\[
  \frac{d}{dt}\left(\frac{\partial T}{\partial\dot{q}}\right)
  - \frac{\partial T}{\partial q}
  = -\frac{\partial V}{\partial q}
\]
\[
  \frac{d}{dt}\left(\frac{\partial T}{\partial\dot{q}}\right)
  - \frac{\partial T}{\partial q}
  + \frac{\partial V}{\partial q} = 0
\]
\begin{equation}\label{eq:el-EL-temp02}
  \frac{d}{dt}\left(\frac{\partial T}{\partial\dot{q}}\right)
  - \frac{\partial (T-V)}{\partial q}
  = 0
\end{equation}

El potencial puede depender de las coordenadas $q$ e incluso del tiempo, pero no depende de la
velocidad $\dot{q}$. Por tanto 
\[
  \frac{\partial V}{\partial\dot{q}} = 0
\]

Por tanto, podemos añadirlo a~\ref{eq:el-EL-temp02}
\[
  \frac{d}{dt}\left(\frac{\partial (T-V)}{\partial\dot{q}}\right)
  - \frac{\partial(T-V)}{\partial q}
  = 0
\]

Si llamamos \emph{lagrangiana}, a $\symcal{L} = T - V$, las ecuaciones de Euler-Lagrange
quedarían
\begin{equation}\label{eq:el-EL-segunda-version}
  \frac{d}{dt}\left(\frac{\partial \symcal{L}}{\partial\dot{q}}\right)
  - \frac{\partial \symcal{L}}{\partial q}
  = 0
\end{equation}
que son válidas para fuerzas que derivan de un potencial que no dependa de la velocidad,
que son las que tienen mayor interés en Física Teórica.
En realidad se trata de una única ecuación, pero la cosa cambia cuando hay varios grados
de libertad, una ecuación por cada uno.

Sólo nos queda explicar cómo se generalizaría en estas otras condiciones,
$N$ partículas y $n$ grados de libertad.
Por tener el sistema $N$ partículas, la energía cinética y el potencial serían
\begin{align*}
  T &= \frac{1}{2}\sum_{i=1}^N m_i v_i^2\\
  V &= \sum_{i=1}^N V_i
\end{align*}

En cuanto a los $n$ grados de libertad, habrá $n$ ecuaciones, una por cada coordenada
generalizada.
La primera variante es
\begin{equation}
  \frac{d}{dt}\left(\frac{\partial T}{\partial\dot{q_j}}\right) - \frac{\partial T}{\partial q_j}
  = \sum_{i=1}^N\vvv{F}_i\cdot\frac{\partial\vvv{r}_i}{\partial q_j}
  \hspace{2em}
  (j = 1, \cdots, n)
\end{equation}

Si las fuerzas provienen de un potencial
\begin{equation}
  \frac{d}{dt}\left(\frac{\partial\symcal{L}}{\partial\dot{q_j}}\right) -
  \frac{\partial\symcal{L}}{\partial q_j}
  = 0
  \hspace{2em}
  (j = 1, \cdots, n)
\end{equation}

\subsection{Ejemplo}
A continuación aplicaremos las ecuaciones de Euler-Lagrange a un sistema con dos grados
de libertad.
El ejemplo que veremos hoy es lo suficientemente sencillo, pero aún así interesante para ver
aspectos que pueden ser útiles en problemas más complicados.
También se adjuntará un pequeño programa en $MATLAB$ para ver como se movería el sistema.

El sistema consiste en una masa $M$ que se puede mover horizontalmente, sin rozamiento,
a lo largo del eje $x$. Esta a su vez, está unida a otra masa $m$ por medio de una varilla
rígida y sin masa, de longitud $L$. La varilla puede formar un ángulo $\theta$ variable
con respecto a la vertical. En la figura~\ref{fig:el-EL-pendulo-Mm} se puede ver un esquema del
sistema.
Nuestro objetivo es encontrar las ecuaciones del movimiento y resolverlas numéricamente.

\begin{figure}[ht]
  % Escala
  \def\scl{.9}
  \centering
  \begin{subfigure}{.415\textwidth}
  % Ejes x e y
  \pgfmathsetmacro{\LXDCHA}{2.9}
  \pgfmathsetmacro{\LXIZDA}{-1.0}
  \pgfmathsetmacro{\LYARRIBA}{.6}
  \pgfmathsetmacro{\LYABAJO}{-2.5}
  % Masa M
  \pgfmathsetmacro{\XM}{1.0}
  \pgfmathsetmacro{\MRADIO}{2.2}
  % Barra rígida
  \pgfmathsetmacro{\LBARRA}{2.3}
  \pgfmathsetmacro{\ANGBARRA}{30}
  % Ángulo
  \pgfmathsetmacro{\RADIOANGULO}{.9}
  % Masa m
  \pgfmathsetmacro{\mRADIO}{1.5}
  % Línea auxiliar
  %\pgfmathsetmacro{\LAUX}{\LBARRA * cos(\ANGBARRA)}
  \pgfmathsetmacro{\LAUX}{\LBARRA}
  % FONDO
  \pgfmathsetmacro{\BGTOP}{1.4}
  \pgfmathsetmacro{\BGBOTTOM}{-3.0}
  \pgfmathsetmacro{\BGLEFT}{-1.6}
  \pgfmathsetmacro{\BGRIGHT}{3.7}
  % 
  \centering
  \begin{tikzpicture}[%
    scale=\scl,
    baseline,
    every node/.style={black,font=\small},
    origen/.style={fill=black, draw=black, radius=.8pt},
    eje/.style={-{>}, black!50, line width=.8pt},
    masa M/.style={fill=red, draw=black, radius=\MRADIO pt},
    masa m/.style={fill=red, draw=black, radius=\mRADIO pt},
    desplazamiento M/.style={-{Latex[length=3pt, width=2.5pt]}, shorten >=5pt, shorten <=2pt, green!55!black},
    angulo/.style={%
      -{Latex[length=3pt, width=2.5pt, bend]}, shorten >=1.5pt, shorten <=1pt,
      draw=green!55!black
    },
    barra/.style={black,line width=1.0pt},
    linea auxiliar/.style={black!15, ultra thin},
    background/.style={%
      line width=\bgborderwidth,
      draw=\bgbordercolor,
      fill=\bgcolor,
    },
    ]
    % COORDENADAS
    % Origen de coordenadas
    \coordinate (O) at (0,0);
    % Ejes
    \coordinate (ejexdcha) at (\LXDCHA, 0);
    \coordinate (ejexizda) at (\LXIZDA, 0);
    \coordinate (ejeyarriba) at (0,\LYARRIBA);
    \coordinate (ejeyabajo) at (0,\LYABAJO);
    % M
    \coordinate (M) at (\XM, 0);
    % Barra
    \coordinate (inibarra) at (M);
    \path (inibarra) -- +(-90+\ANGBARRA:\LBARRA) coordinate (finbarra);
    % m
    \coordinate (m) at (finbarra);
    % Fondo
    \coordinate (bgtop) at (0,\BGTOP);
    \coordinate (bgbottom) at (0,\BGBOTTOM);
    \coordinate (bgleft) at (\BGLEFT,0);
    \coordinate (bgright) at (\BGRIGHT,0);
    % DIBUJO
    % Ángulo
    \draw[angulo] ($(\XM,0)-(0,\RADIOANGULO)$)
    arc [start angle=270, end angle=270+\ANGBARRA, radius=\RADIOANGULO]
    node[above=1pt,pos=.33,green!50!black]  {\footnotesize$\theta$};
    % Línea auxiliar
    \draw[linea auxiliar] (M) -- +(down:\LAUX);
    % Ejes
    \draw[eje] (ejexizda) -- (ejexdcha) node[right] {$x$};
    \draw[eje] (ejeyabajo) -- (ejeyarriba) node[above] {$y$};
    % Desplazamiento de M
    \draw[desplazamiento M]
    ($(O) + (0, 4pt)$) -- node[above, pos=.4, green!50!black] {$x$} ($(M) + (0,4pt)$);
    % Barra
    \draw[barra] (inibarra) -- node[right=1pt, pos=.5] {$L$} (finbarra);
    % Masas
    \filldraw[masa M] (M) circle node[above right=1.6pt and 0pt] {$M$};
    \filldraw[masa m] (m) circle node[below right=1.0pt and -2pt] {$m$};
    % Origen
    \filldraw[origen] (O) circle;
    % YELLOW BACKGROUND
    \begin{scope}[on background layer]
      \node [background, fit= (bgleft) (bgright) (bgtop) (bgbottom)] {};
    \end{scope}
  \end{tikzpicture}
  \caption{En verde se representan las coordenadas generalizadas.\\}
  \label{fig:el-EL-pendulo-Mm}
\end{subfigure}
\hspace{2em}
\begin{subfigure}{.415\textwidth}
  % Ejes x e y
  \pgfmathsetmacro{\LXDCHA}{2.9}
  \pgfmathsetmacro{\LXIZDA}{-1.0}
  \pgfmathsetmacro{\LYARRIBA}{.6}
  \pgfmathsetmacro{\LYABAJO}{-2.5}
  % Masa M
  \pgfmathsetmacro{\XM}{1.0}
  \pgfmathsetmacro{\MRADIO}{2.2}
  % Barra rígida
  \pgfmathsetmacro{\LBARRA}{2.3}
  \pgfmathsetmacro{\ANGBARRA}{30}
  % Ángulo
  \pgfmathsetmacro{\RADIOANGULO}{.9}
  % Masa m
  \pgfmathsetmacro{\mRADIO}{1.5}
  % Línea auxiliar
  %\pgfmathsetmacro{\LAUX}{\LBARRA * cos(\ANGBARRA)}
  \pgfmathsetmacro{\LAUX}{\LBARRA}
  % FONDO
  \pgfmathsetmacro{\BGTOP}{1.4}
  \pgfmathsetmacro{\BGBOTTOM}{-3.0}
  \pgfmathsetmacro{\BGLEFT}{-1.6}
  \pgfmathsetmacro{\BGRIGHT}{3.7}
  % 
  \centering
  \begin{tikzpicture}[%
    scale=\scl,
    baseline,
    every node/.style={black, font=\small},
    origen/.style={fill=black, draw=black, radius=.8pt},
    eje/.style={-{>}, black!50, line width=.8pt},
    masa M/.style={fill=red, draw=black, radius=\MRADIO pt},
    masa m/.style={fill=red, draw=black, radius=\mRADIO pt},
    vector posicion/.style={-{Latex[length=7pt,width=4.5pt]}, shorten >=2pt, shorten <=1pt,
      line width=1.4pt, red!80!black},
    vector/.style={-{Latex[length=7pt,width=4.5pt]}, shorten >=3pt, shorten <=1pt,
      line width=1.4pt, black},
    barra/.style={black,line width=1.0pt},
    linea auxiliar/.style={black!15, ultra thin},
    background/.style={%
      line width=\bgborderwidth,
      draw=\bgbordercolor,
      fill=\bgcolor,
    },
    ]
    % COORDENADAS
    % Origen de coordenadas
    \coordinate (O) at (0,0);
    % Ejes
    \coordinate (ejexdcha) at (\LXDCHA, 0);
    \coordinate (ejexizda) at (\LXIZDA, 0);
    \coordinate (ejeyarriba) at (0,\LYARRIBA);
    \coordinate (ejeyabajo) at (0,\LYABAJO);
    % M
    \coordinate (M) at (\XM, 0);
    % Barra
    \coordinate (inibarra) at (M);
    \path (inibarra) -- +(-90+\ANGBARRA:\LBARRA) coordinate (finbarra);
    % m
    \coordinate (m) at (finbarra);
    % Fondo
    \coordinate (bgtop) at (0,\BGTOP);
    \coordinate (bgbottom) at (0,\BGBOTTOM);
    \coordinate (bgleft) at (\BGLEFT,0);
    \coordinate (bgright) at (\BGRIGHT,0);
    % DIBUJO
    % Ejes
    \draw[eje] (ejexizda) -- (ejexdcha) node[right] {$x$};
    \draw[eje] (ejeyabajo) -- (ejeyarriba) node[above] {$y$};
    % Vectores posición
    \draw[vector posicion]
    (O) -- node[above, pos=.4, red!80!black] {$\vvv{r}_M$} (M);
    \draw[vector posicion, shorten >=3pt]
    (O) -- node[below left=-4pt and -2pt, red!80!black] {$\vvv{r}_m$} (m);
    % Vector barra
    \draw[vector] (inibarra) -- node[right=1pt, pos=.5] {$\vvv{L}$} (finbarra);
    % Masas
    \filldraw[masa M] (M) circle node[above=1.6pt] {$M$};
    \filldraw[masa m] (m) circle node[below right=1.0pt and -2pt] {$m$};
    % Origen
    \filldraw[origen] (O) circle;
    % YELLOW BACKGROUND
    \begin{scope}[on background layer]
      \node [background, fit= (bgleft) (bgright) (bgtop) (bgbottom)] {};
    \end{scope}
  \end{tikzpicture}
  \caption{En rojo se representan las posiciones de las dos masas. En negro el vector que
    representa la barra.}
  \label{fig:el-EL-pendulo-Mm-posiciones}
\end{subfigure}
\caption{El sistema de la figura está formado por una masa $M$, que puede deslizar sin fricción
  a lo largo del eje $x$, unida a otra masa $m$ por medio de una barra rígida de masa
  despreciable y longitud $L$. La barra puede girar alrededor de $M$, desviándose un ángulo
  variable con respecto a la vertical.}
\end{figure}

Las coordenadas generalizadas que usaremos son la coordenada $q_1 = x$ de la masa $M$ y el ángulo
$q_2 = \theta$ que forma la barra con la vertical. Estas dos coordenadas\footnotemark{}
determinan completamente la geometría del sistema. Así, el sistema tiene dos grados de libertad.
\footnotetext{Utilizaremos los símbolos $x$ y $\theta$ en lugar de $q_1$ y $q_2$, pues son símbolos
  más descriptivos.}

Los vectores de posición de las dos masas, $\vvv{r}_M$ y $\vvv{r}_m$, y el vector $\vvv{L}$ que
representa la barra, están representados en la figura~\ref{fig:el-EL-pendulo-Mm-posiciones}.
Estos vectores valen
\begin{align}
  &\vvv{L} = (L\sin\theta, -L\cos\theta)\\
  &\vvv{r}_M = (x, 0)\\
  &\vvv{r}_m = \vvv{r}_M + \vvv{L}
    = (x,0) + (L\sin\theta, -L\cos\theta) = (x+L\sin\theta, -L\cos\theta)
\end{align}

Calculamos la derivada con respecto del tiempo del vector $\vvv{L}$
\[
  \vvv{\dot{L}}
  = \left(L\dot{\theta}\cos\theta, L\dot{\theta}\sin\theta\right)
  = L\dot{\theta}\left(\cos\theta,\sin\theta\right)
\]

Queremos calcular la energía cinética, que depende del cuadrado de las velocidades
de las masas. Empezamos calculando los vectores velocidad $\vvv{v}_M = \vvv{\dot{r}}_M$ y
$\vvv{v}_m = \vvv{{\dot{r}}_m}$
\begin{align}
  \vvv{v}_M &= \vvv{\dot{r}}_M = \left(\dot{x}, 0\right)\\
  \vvv{v}_m &= \vvv{\dot{r}}_m = \vvv{\dot{r}}_M + \vvv{\dot{L}}
              = \left(\dot{x},0\right) + L\dot{\theta}\left(\cos\theta,\sin\theta\right)
              = \left(\dot{x}+L\dot{\theta}\cos\theta, L\dot{\theta}\sin\theta\right)
\end{align}

A continuación calculamos el cuadrado de las velocidades.
Empezamos por $v_M^2$
\begin{equation}
  v_M^2 = \dot{x}^2
\end{equation}
y seguimos por $v_m^2$ que es algo más elaborado
\begin{align*}
  v_m^2
  &= \vvv{v}_m\cdot\vvv{v}_m
    = \left(\dot{x}+L\dot{\theta}\cos\theta, L\dot{\theta}\sin\theta\right)
    \cdot\left(\dot{x}+L\dot{\theta}\cos\theta, L\dot{\theta}\sin\theta\right)\\
  &= \left(\dot{x}+L\dot{\theta}\cos\theta\right)^2
    + \left(L\dot{\theta}\sin\theta\right)^2\\
  &= \dot{x}^2
    + 2L\dot{x}\dot{\theta}\cos\theta
    + L^2\dot{\theta}^2\cos^2\theta
    + L^2\dot{\theta}^2\sin^2\theta\\
\end{align*}

Simplificando la última expresión, nos queda
\begin{equation}
  v_m^2 = \dot{x}^2 + 2L\dot{x}\dot{\theta}\cos\theta + L^2\dot{\theta}^2
\end{equation}


La energía cinética del sistema es la suma de las energías cinéticas de las partículas
\[
  T = \frac{1}{2} M v_M^2 + \frac{1}{2} m v_m^2
  = \dfrac{1}{2}M\dot{x}^2
  + \dfrac{1}{2}m
  \left(\dot{x}^2 + 2L\dot{x}\dot{\theta}\cos\theta + L^2\dot{\theta}^2
  \right)
\]

Reorganizando términos, la energía cinética queda
\begin{equation}
  T = \dfrac{1}{2}(M+m)\dot{x}^2 + \dfrac{1}{2}mL^2\dot{\theta}^2 + mL\dot{x}\dot{\theta}\cos\theta
\end{equation}

La energía potencial $V$ del sistema es la suma de las energías potenciales de cada partícula
$V = V_M + V_m$. Situamos el origen para medir las alturas en $y=0$, entonces
\begin{equation}
  V = Mg\cdot 0 + (-mgL\cos\theta) = -mgL\cos\theta
\end{equation}

La función lagrangiana $\symcal{L}$ es
\[
  \symcal{L} = T - V
  = \dfrac{1}{2}(M+m)\dot{x}^2 + \dfrac{1}{2}mL^2\dot{\theta}^2 + mL\dot{x}\dot{\theta}\cos\theta
  - (-mgL\cos\theta)
\]

La lagrangiana resulta
\begin{equation}
  \symcal{L}
  = \dfrac{1}{2}(M+m)\dot{x}^2 + \dfrac{1}{2}mL^2\dot{\theta}^2 + mL\dot{x}\dot{\theta}\cos\theta
  + mg\cos\theta
\end{equation}

Ahora calculamos las derivadas parciales de la lagrangiana relacionadas con $x$
\begin{align}
  \frac{\partial\symcal{L}}{\partial x} &= 0\\
  \frac{\partial\symcal{L}}{\partial\dot{x}} &= (M+m)\dot{x} + mL\dot{\theta}\cos\theta
\end{align}
y las relacionadas con $\theta$
\begin{align}
  \frac{\partial\symcal{L}}{\partial\theta} &= -mL\dot{x}\dot{\theta}\sin\theta-mgL\sin\theta\\
  \frac{\partial\symcal{L}}{\partial\dot{\theta}} &= mL^2\dot{\theta} + mL\dot{x}\cos\theta
\end{align}

Resolvemos las ecuaciones de Euler-Lagrange:
\begin{itemize}
\item La relacionada con $x$
  \[
    \frac{d}{dt}\left(\frac{\partial\symcal{L}}{\partial\dot{x}}\right)
    = \frac{\partial\symcal{L}}{\partial x}
  \]
  Cuando la lagrangiana no depende de una variable generalizada ($x$ en este caso)
  se dice que es una \emph{variable cíclica}.
  Por ahora, nos tomaremos esto como una curiosidad. En su momento veremos su relación
  con las leyes de conservación
  \[
    \frac{d}{dt}\left[(M+m)\dot{x}+mL\dot{\theta}\cos\theta\right] = 0
  \]
  Esto implica que la cantidad que se deriva se mantiene constante
  \begin{equation}\label{eq:el-EL-pendulo-Mm-ecuacion-x}
    (M+m)\dot{x}+mL\dot{\theta}\cos\theta = A
  \end{equation}
  
\item La relacionada con $\theta$
  \[
    \frac{d}{dt}\left(\frac{\partial\symcal{L}}{\partial\dot{\theta}}\right)
    = \frac{\partial\symcal{L}}{\partial\theta}
  \]
  \[
    \frac{d}{dt}\left(mL^2\dot{\theta} + mL\dot{x}\cos\theta\right)
    = -mL\dot{x}\dot{\theta}\sin\theta-mg\sin\theta
  \]
  \[
    mL^2\ddot{\theta} + mL\left(\ddot{x}\cos\theta - \dot{x}\dot{\theta}\sin\theta\right)
    = -mL\dot{x}\dot{\theta}\sin\theta-mgL\sin\theta
  \]

  Simplificamos y reorganizamos términos, obtenemos la ecuación del movimiento con respecto a la
  variable $\theta$
  \begin{equation}\label{eq:el-EL-pendulo-Mm-ecuacion-theta}
    \ddot{\theta} = -\frac{\ddot{x}}{L}\cos\theta - \frac{g}{L}\sin\theta
  \end{equation}
\end{itemize}

Vemos que las dos ecuaciones del movimiento, \ref{eq:el-EL-pendulo-Mm-ecuacion-theta}
y \ref{eq:el-EL-pendulo-Mm-ecuacion-theta} están acopladas, lo que significa que en
las dos se encuentran las variables $x$ y $\theta$ o sus derivadas temporales.

Cuando las ecuaciones están acopladas no se pueden resolver la mayor parte de las veces.
Como consecuencia, hemos realizado una simulación numérica, con los siguientes
valores iniciales
\begin{align*}
  \dot{x} &= 0\\
  \dot{\theta} &= 0
\end{align*}





%janr







%%% Local Variables:
%%% mode: latex
%%% TeX-engine: luatex
%%% TeX-master: "../mecanicateorica.tex"
%%% End:

% LaTeX-command: "lualatex --shell-escape
