% momento-fuerza.tex
% Momento de una fuerza
%
% Copyright (C) 2022-2025 José A. Navarro Ramón <janr.devel@gmail.com>
% 1) Código LuaLatex:
%    Licencia GPL-2.
% 2) Producto en pdf, postscript, etc.:
%    Licencia Creative Commons Recognition Share alike. (CC-BY-SA)

\chapter{Momento de una fuerza}
\label{chapt:ap-momento-fuerza}

Vamos a explicar muy brevemente cómo calcular el momento de una fuerza $\vvv{F}$.
  Primero hemos de elegir un punto arbitrario llamado centro, $O$, y denotaremos por
  $\vvv{r}$ al vector que va del centro al punto de aplicación de la fuerza,
  ver figura~\ref{fig:el-ap-rF}.

  El módulo del momento con respecto al centro elegido es el módulo de un producto vectorial
  \begin{equation}\label{eq:ap-modmomang}
    |\vvv{M}| = |\vvv{r}\times\vvv{F}| = |\vvv{r}| |\vvv{F}| \sin\alpha
  \end{equation}
  donde $\alpha$ es el ángulo que forman los vectores $\vvv{r}$ y $\vvv{F}$. Para visualizar
  $\alpha$ basta con fijarse en el ángulo que forma la recta de acción de $\vvv{r}$ cuando
  sobrepasa al punto de aplicación de la fuerza, como se puede apreciar en la
  figura~\ref{fig:el-ap-rF-alfa}.

  \begin{figure}[ht]
    \def\scl{1.05}
    %
    \pgfmathsetmacro{\XFUERZA}{1.8}
    \pgfmathsetmacro{\YFUERZA}{.8}
    \pgfmathsetmacro{\LFUERZA}{1.8}
    \pgfmathsetmacro{\ANGFUERZA}{340}
    % Fondo
    \pgfmathsetmacro{\BGTOP}{1.8}
    \pgfmathsetmacro{\BGBOTTOM}{-.8}
    \pgfmathsetmacro{\BGLEFT}{-.8}
    \pgfmathsetmacro{\BGRIGHT}{4.1}    
    %
    \centering
    \begin{subfigure}[b]{.45\textwidth}
    %
    %\centering
    \begin{tikzpicture}[%
      scale=\scl,
      baseline,
      every node/.style={black,font=\small},
      angulo/.style={draw=green!50!black, fill=green!20},
      r/.style={-{Latex[round, width=4pt]}, shorten >=1pt, line width=1pt, black},
      F/.style={-{Latex[round, width=4pt]}, line width=1pt, green!60!black},
      punto fuerza/.style={fill=green!60!black, draw=green!60!black, radius=.8pt},
      centro/.style={fill=black, draw=black, radius=.8pt},
      punto/.style={fill=black, draw=black, radius=.8pt},
      background/.style={%
        line width=\bgborderwidth,
        draw=\bgbordercolor,
        fill=\bgcolor,
      },
      ]
      % COORDENADAS
      \coordinate (O) at (0,0);
      \coordinate (OF) at (\XFUERZA,\YFUERZA);
      % 
      \coordinate (btop) at (0,\BGTOP);
      \coordinate (bbottom) at (0,\BGBOTTOM);
      \coordinate (bleft) at (\BGLEFT,0);
      \coordinate (bright) at (\BGRIGHT,0);
      % DIBUJO
      % Vector posición
      \draw[r] (O) -- node[above] {$\vvv{r}$} (OF);
      % Centro
      \filldraw[centro] (O) circle;
      \node[below left] at (O) {$O$};
      % Fuerza
      \draw[F]
      (OF) -- node[above right=-2pt and -1pt,,green!60!black] {$\vvv{F}$} +(\ANGFUERZA:\LFUERZA);
      \filldraw[punto fuerza] (OF) circle;
       FONDO AMARILLO
      \begin{scope}[on background layer]
        \node [background, fit= (bleft) (bright) (btop) (bbottom)] {};
      \end{scope}
    \end{tikzpicture}
    \caption{Para calcular el momento de una fuerza con respecto de un punto llamado
      \emph{centro} ($O$), se comienza trazando el vector $\vvv{r}$ que va desde el centro
      hasta el punto de aplicación de la fuerza.}
    \label{fig:el-ap-rF}
  \end{subfigure}
  \hspace{2em}
  \begin{subfigure}[b]{.45\textwidth}
    % 
    \pgfmathsetmacro{\LFUERZA}{1.8}
    \pgfmathsetmacro{\ANGFUERZA}{-20}
    %
    \pgfmathsetmacro{\ANGRECTA}{25}
    \pgfmathsetmacro{\LR}{2}
    \pgfmathsetmacro{\LRECTA}{1.4}
    %
    \pgfmathsetmacro{\RADIOANGULO}{2}
    %
    \pgfmathsetmacro{\LRECTA}{1.4}
    % Fondo
    \pgfmathsetmacro{\BGTOP}{1.8}
    \pgfmathsetmacro{\BGBOTTOM}{-.8}
    \pgfmathsetmacro{\BGLEFT}{-.8}
    \pgfmathsetmacro{\BGRIGHT}{4.1}    
    %
    %\centering
    \begin{tikzpicture}[%
      scale=\scl,
      baseline,
      every node/.style={black,font=\small},
      angulo/.style={{Latex[round,width=4pt]}-,draw=green!50!black, fill=green!20},
      recta/.style={ultra thin, black!30},
      r/.style={-{Latex[round, width=4pt]}, shorten >=1pt, line width=1pt, black},
      F/.style={-{Latex[round, width=4pt]}, line width=1pt, green!60!black},
      punto fuerza/.style={fill=green!60!black, draw=green!60!black, radius=.8pt},
      centro/.style={fill=black, draw=black, radius=.8pt},
      punto/.style={fill=black, draw=black, radius=.8pt},
      background/.style={%
        line width=\bgborderwidth,
        draw=\bgbordercolor,
        fill=\bgcolor,
      },
      ]
      % COORDENADAS
      \coordinate (O) at (0,0);
      %\coordinate (OF) at (\XFUERZA,\YFUERZA);
      \coordinate (OF) at ($(O) + (\ANGRECTA:\LR)$);
      % 
      \coordinate (btop) at (0,\BGTOP);
      \coordinate (bbottom) at (0,\BGBOTTOM);
      \coordinate (bleft) at (\BGLEFT,0);
      \coordinate (bright) at (\BGRIGHT,0);
      % DIBUJO
      % Ángulo
      \filldraw[angulo] (OF) -- +(\ANGFUERZA:\RADIOANGULO em)
      arc [start angle=\ANGFUERZA, end angle=\ANGRECTA, radius=\RADIOANGULO em] -- cycle;
      \node[above right=-5.1pt and 7pt] at (OF) {$\alpha$};
      \draw[angulo] ($(OF) + (\ANGFUERZA:\RADIOANGULO em)$)
       arc [start angle=\ANGFUERZA, end angle=\ANGRECTA,  radius=\RADIOANGULO em];
      % Recta de acción
      \draw[recta] (OF) -- +(\ANGRECTA:\LRECTA);
      % Vector posición
      \draw[r] (O) -- node[above] {$\vvv{r}$} (OF);
      % Centro
      \filldraw[centro] (O) circle;
      \node[below left] at (O) {$O$};
      % Fuerza
      \draw[F] (OF) -- node[below,green!60!black] {$\vvv{F}$} +(\ANGFUERZA:\LFUERZA);
      \filldraw[punto fuerza] (OF) circle;
      % FONDO AMARILLO
      \begin{scope}[on background layer]
        \node [background, fit= (bleft) (bright) (btop) (bbottom)] {};
      \end{scope}
    \end{tikzpicture}
    \caption{El módulo del momento de la fuerza se puede hallar multiplicando $|\vvv{r}|$
      por $|\vvv{F}|$ y por $\sin\alpha$. El ángulo es útil para hallar el sentido del momento
      cuando las fuerzas son coplanares.}
    \label{fig:el-ap-rF-alfa}
  \end{subfigure}
  \caption{}
\end{figure}

Para hallar el sentido del momento podemos fijarnos en el del ángulo de giro desde $\vvv{r}$ hasta
$\vvv{F}$. En el caso de la fuerza de la figura se observa que el sentido es negativo (horario),
por lo que el momento es perpendicular al plano formado por $\vvv{r}$ y $\vvv{F}$ y está
dirigido hacia dentro.

Cuando todas las fuerzas del sistema sean coplanares nos bastaría el signo del ángulo para
obtener el sentido del momento de cada fuerza y poder componer momentos.

%Por supuesto, lo anterior sirve si conocemos $|\vvv{r}|$, $|\vvv{F}|$ y el ángulo $\alpha$ que
%forman estos dos vectores.
%Pero, dependiendo de los datos que tengamos, podría ser más cómodo descomponer la fuerza en una
%componente paralela a $\vvv{r}$ y otra perpendicular
En la figura~\ref{fig:el-ap-r-FF} se ha descompuesto la fuerza $\vvv{F}$ en dos componentes, una
paralela y otra perpendicular al vector $\vvv{r}$
\[
  \vvv{F} = \vvv{F}_\parallel + \vvv{F}_\perp
\]

\begin{figure}[ht]
  \def\scl{1.05}
  % 
  \centering
  \begin{subfigure}[b]{.45\textwidth}
    % 
    \pgfmathsetmacro{\LFUERZA}{1.8}
    \pgfmathsetmacro{\ANGFUERZA}{-20}
    % 
    \pgfmathsetmacro{\ANGRECTA}{25}
    \pgfmathsetmacro{\LR}{2}
    \pgfmathsetmacro{\LRECTA}{1.4}
    % 
    \pgfmathsetmacro{\RADIOANGULO}{2}
    %
    \pgfmathsetmacro{\LRECTA}{1.6}
    % Componentes paralela y perpendicular de la fuerza
    \pgfmathsetmacro{\FPARALELA}{\LFUERZA * cos(\ANGFUERZA-\ANGRECTA)}
    \pgfmathsetmacro{\FPERPENDICULAR}{\LFUERZA * sin(\ANGFUERZA-\ANGRECTA)}
    % Fondo
    \pgfmathsetmacro{\BGTOP}{1.8}
    \pgfmathsetmacro{\BGBOTTOM}{-.8}
    \pgfmathsetmacro{\BGLEFT}{-.8}
    \pgfmathsetmacro{\BGRIGHT}{4.1}    
    %
    %\centering
    \begin{tikzpicture}[%
      scale=\scl,
      baseline,
      every node/.style={black,font=\small},
      angulo/.style={{Latex[round,width=3pt]}-, shorten >=2pt, draw=green!50!black, fill=green!20},
      angulo auxiliar/.style={{Latex[round,width=3pt]}-, shorten >=2pt, draw=black!50, fill=black!10},
      recta/.style={ultra thin, black!30},
      r/.style={-{Latex[round, width=4pt]}, shorten >=1pt, line width=1pt, black},
      fuerza/.style={-{Latex[round, width=4pt]}, line width=1pt, black!30},
      componente fuerza/.style={-{Latex[round, width=4pt]}, line width=1pt, green!60!black},
      linea auxiliar/.style={ultra thin, light gray},
      punto fuerza/.style={fill=green!60!black, draw=green!60!black, radius=.8pt},
      centro/.style={fill=black, draw=black, radius=.8pt},
      punto/.style={fill=black, draw=black, radius=.8pt},
      background/.style={%
        line width=\bgborderwidth,
        draw=\bgbordercolor,
        fill=\bgcolor,
      },
      ]
      % COORDENADAS
      \coordinate (O) at (0,0);
      % \coordinate (OF) at (\XFUERZA,\YFUERZA);
      % Origen de la fuerza
      \coordinate (OF) at ($(O) + (\ANGRECTA:\LR)$);
      % Extremo de la fuerza
      \path (OF) -- +(\ANGFUERZA:\LFUERZA) coordinate (F);
      % 
      \coordinate (btop) at (0,\BGTOP);
      \coordinate (bbottom) at (0,\BGBOTTOM);
      \coordinate (bleft) at (\BGLEFT,0);
      \coordinate (bright) at (\BGRIGHT,0);
      % Semirrecta paralela a la recta de acción
      \path (OF) -- +(\ANGRECTA:\LRECTA) coordinate (rparalela);
      % Semirrecta perpendicular a la anterior
      \path (OF) -- +(\ANGRECTA-90:\LRECTA) coordinate (rperpendicular);
      % Componente paralela de la fuerza
      \path (OF) -- +(\ANGRECTA:\FPARALELA) coordinate (Fparalela);
      % Componente perpendicular de la fuerza
      \path (OF) -- +(\ANGRECTA-90:\FPARALELA) coordinate (Fperpendicular);
      % DIBUJO
      % Ángulo
      \filldraw[angulo auxiliar] (OF) -- +(\ANGFUERZA:\RADIOANGULO em)
      arc [start angle=\ANGFUERZA, end angle=\ANGRECTA, radius=\RADIOANGULO em] -- cycle;
      \node[above right=-5.1pt and 7pt] at (OF) {\textcolor{black!60}{$\alpha$}};
      \draw[angulo auxiliar] ($(OF) + (\ANGFUERZA:\RADIOANGULO em)$)
       arc [start angle=\ANGFUERZA, end angle=\ANGRECTA,  radius=\RADIOANGULO em];
      % Recta paralela
      \draw[recta] (OF) -- (rparalela);
      % Recta perpendicular
      \draw[recta] (OF) -- (rperpendicular);
      % Vector posición
      \draw[r] (O) -- node[above] {$\vvv{r}$} (OF);
      % Centro
      \filldraw[centro] (O) circle;
      \node[below left] at (O) {$O$};
      % Línea auxiliar para componente paralela de la fuerza
      \draw[ultra thin, lightgray] (F) -- (Fparalela);
      % Línea auxiliar para componente perpendicular de la fuerza
      \draw[ultra thin, lightgray] (F) -- (Fperpendicular);
      % Fuerza
      \draw[fuerza] (OF) -- node[below left=1pt and -10pt,black!30] {$\vvv{F}$} (F);
      % Componente paralela de la fuerza
      \draw[componente fuerza]
      (OF) -- node[above left=-2pt and -3pt, green!60!black] {$\vvv{F}_\parallel$} (Fparalela);
      % Componente perpendicular de la fuerza
      \draw[componente fuerza]
      (OF) -- node[below left=-2pt and -3pt, green!60!black] {$\vvv{F}_\perp$} (Fperpendicular);
      % Punto de aplicación de la fuerza
      \filldraw[punto fuerza] (OF) circle;
      % FONDO AMARILLO
      \begin{scope}[on background layer]
        \node [background, fit= (bleft) (bright) (btop) (bbottom)] {};
      \end{scope}
    \end{tikzpicture}
    \caption{$|\vvv{M}| = |\vvv{r}|\, |\vvv{F_\perp}|$}
    \label{fig:el-ap-r-FF}
  \end{subfigure}
  \hspace{2em}
  \begin{subfigure}[b]{.45\textwidth}
    % 
    \pgfmathsetmacro{\LFUERZA}{1.8}
    \pgfmathsetmacro{\ANGFUERZA}{-20}
    % 
    \pgfmathsetmacro{\ANGRECTA}{25}
    \pgfmathsetmacro{\LR}{2}
    \pgfmathsetmacro{\LRECTA}{1.4}
    %
    \pgfmathsetmacro{\RADIOANGULO}{2}
    %
    \pgfmathsetmacro{\LRECTAPARALELA}{2.1}
    \pgfmathsetmacro{\LRECTAPERPENDICULAR}{1.7}
    % Componentes paralela y perpendicular de r
    \pgfmathsetmacro{\RPERPENDICULAR}{\LR * cos(\ANGFUERZA-\ANGRECTA)}

    % Fondo
    \pgfmathsetmacro{\BGTOP}{1.8}
    \pgfmathsetmacro{\BGBOTTOM}{-.8}
    \pgfmathsetmacro{\BGLEFT}{-.8}
    \pgfmathsetmacro{\BGRIGHT}{4.1}    
    %
    %\centering
    \begin{tikzpicture}[%
      scale=\scl,
      baseline,
      every node/.style={black,font=\small},
      angulo/.style={{Latex[round,width=3pt]}-, shorten >=2pt, draw=green!50!black, fill=green!20},
      angulo auxiliar/.style={{Latex[round,width=3pt]}-, shorten >=2pt, draw=black!50, fill=black!10},
      recta/.style={ultra thin, black!30},
      r/.style={-{Latex[round, width=4pt]}, shorten >=1pt, line width=1pt, black!30},
      r perpendicular/.style={-{Latex[round, width=4pt]}, shorten >=1pt, line width=1pt, black},
      fuerza/.style={-{Latex[round, width=4pt]}, line width=1pt, green!60!black},
      linea auxiliar/.style={ultra thin, lightgray},
      punto fuerza/.style={fill=green!60!black, draw=green!60!black, radius=.8pt},
      centro/.style={fill=black, draw=black, radius=.8pt},
      punto/.style={fill=black, draw=black, radius=.8pt},
      background/.style={%
        line width=\bgborderwidth,
        draw=\bgbordercolor,
        fill=\bgcolor,
      },
      ]
      % COORDENADAS
      \coordinate (O) at (0,0);
      % \coordinate (OF) at (\XFUERZA,\YFUERZA);
      % Origen de la fuerza
      \coordinate (OF) at ($(O) + (\ANGRECTA:\LR)$);
      % Extremo de la fuerza
      \path (OF) -- +(\ANGFUERZA:\LFUERZA) coordinate (F);
      %
      \coordinate (btop) at (0,\BGTOP);
      \coordinate (bbottom) at (0,\BGBOTTOM);
      \coordinate (bleft) at (\BGLEFT,0);
      \coordinate (bright) at (\BGRIGHT,0);
      % Semirrecta paralela a la fuerza
      \path (OF) -- +(180+\ANGFUERZA:\LRECTAPARALELA) coordinate (Fparalela);
      % Semirrecta perpendicular a la anterior
      \path (O) -- +(\ANGFUERZA+90:\LRECTAPERPENDICULAR) coordinate (Fperpendicular);
      % DIBUJO
      % Ángulo
      \filldraw[angulo auxiliar] (OF) -- +(\ANGFUERZA:\RADIOANGULO em)
      arc [start angle=\ANGFUERZA, end angle=\ANGRECTA, radius=\RADIOANGULO em] -- cycle;
      \node[above right=-5.1pt and 7pt] at (OF) {\textcolor{black!60}{$\alpha$}};
      \draw[angulo auxiliar] ($(OF) + (\ANGFUERZA:\RADIOANGULO em)$)
       arc [start angle=\ANGFUERZA, end angle=\ANGRECTA,  radius=\RADIOANGULO em];
      % Vector posición
      \draw[r] (O) -- node[below,black!30] {$\vvv{r}$} (OF);
      % Centro
      \filldraw[centro] (O) circle;
      \node[below left] at (O) {$O$};
      % Recta de acción de r
      \draw[recta] (OF) -- +(\ANGRECTA:\LRECTA);
      % Recta de acción de la fuerza
      \draw[linea auxiliar] (OF) -- (Fparalela);
      % Perpendicular que pasa por O
      \draw[linea auxiliar] (O) -- (Fperpendicular);
      % Componente de r perpendicular a la fuerza
      \draw[r perpendicular] (O) -- node[left] {$\vvv{r}_\perp$} (\ANGFUERZA + 90:\RPERPENDICULAR);
      % Fuerza
      \draw[fuerza] (OF) -- node[below left=1pt and -10pt,green!60!black] {$\vvv{F}$} (F);
      % Punto de aplicación de la fuerza
      \filldraw[punto fuerza] (OF) circle;
      % FONDO AMARILLO
      \begin{scope}[on background layer]
        \node [background, fit= (bleft) (bright) (btop) (bbottom)] {};
      \end{scope}
    \end{tikzpicture}
    \caption{$|\vvv{M}| = |\vvv{r_\perp}|\, |\vvv{F}|$}
    %\label{fig:el-ap-rF-alfa}
  \end{subfigure}
  \caption{En esta figura se presentan dos formas equivalentes para calcular el módulo  del momento
    de una fuerza, dependiendo de los datos que se conozcan de la geometría del sistema. El ángulo
    sirve para calcular el sentido del momento de la fuerza cuando las fuerzas del problema sean
    coplanares.}
\end{figure}

Se puede reescrbir la ecuación~\ref{eq:ap-modmomang}, pues
$|\vvv{F}|\sin\alpha = |\vvv{F}_\perp|$
\begin{equation}\label{eq:ap-modulo-F-perp}
  |\vvv{M}| = |\vvv{r}|\,|\vvv{F}|\,\sin\alpha = |\vvv{r}|\,|\vvv{F}_\perp|
\end{equation}

Lo interesante de esto es que la fuerza $\vvv{F}_\parallel$ paralela a $\vvv{r}$ no genera momento,
y por eso no interviene en la expresión anterior. Esta afirmación se puede comprobar intuitivamente
si suponemos que $\vvv{r}$ es una barra rígida que puede girar alrededor del punto fijo $O$;
entonces la componente paralela no haría girar la barra.
Por tanto sólo necesitaríamos conocer la componente de la fuerza perpendicular a $\vvv{r}$.
Esto se puede expresar con algo más de detalle
\begin{align*}
  |\vvv{M}
  &= |\vvv{r}\times\vvv{F}|
    = |\vvv{r}\times(\vvv{F}_\parallel + \vvv{F}_\perp)|
    = |\vvv{r}\times\vvv{F}_\parallel + \vvv{r}\times\vvv{F}_\perp|
    = |\vvv{0} + \vvv{r}\times\vvv{F}_\perp|\\
  &= |\vvv{r}\times\vvv{F}_\perp|
    = |\vvv{r}| |\vvv{F}_\perp| \sin\pi/2
    = |\vvv{r}| |\vvv{F}_\perp|
\end{align*}

También, podríamos reorganizar de otro modo los factores de la
ecuación~\ref{eq:ap-modmomang}
\begin{equation}\label{eq:ap-modulo-r-perp}
  |\vvv{M}|
  = |\vvv{r}|\,|\vvv{F}|\,\sin\alpha
  = |\vvv{F}|\,|\vvv{r}|\,\sin\alpha
  = |\vvv{F}|\,|\vvv{r}_\perp|
\end{equation}
lo que es útil en aquellos ejercicios en los que sea más fácil calcular la
componente perpendicular de $\vvv{r}$ a la recta de acción de $\vvv{F}$.

Otra forma de calcular el momento de la fuerza es
\begin{equation}\label{eq:ap-momento-determinante}
  \vvv{M}
  = \vvv{r}\times\vvv{F}
  =
  \begin{vNiceMatrix}
    \uvec{\i} & \uvec{\j} & \xhat{k}\\
    x & y & z\\
    F_x & F_y & F_z\\
  \end{vNiceMatrix}
\end{equation}
que es útil si conocemos las componentes cartesianas de los vectores
$\vvv{r} = x\uvec{\i} + y\uvec{\j} + z\xhat{k}$ y
$\vvv{F} = F_x\uvec{\i} + F_y\uvec{\j} + F_z\xhat{k}$.









%%% Local Variables:
%%% coding: utf-8
%%% mode: latex
%%% TeX-engine: luatex
%%% TeX-master: "../mecanicateorica.tex"
%%% End:

